% !TeX encoding = UTF-8
% !TeX spellcheck = ru_RU
% !TeX root = lecture7.tex

\documentclass[
    8pt,
    hyperref={pdfencoding=unicode}
    ]{beamer}

\usepackage{polyglossia}   %% загружает пакет многоязыковой вёрстки

\setmainlanguage{russian}  %% устанавливает главный язык документа
\setmainfont{Liberation Serif}
\setromanfont{Liberation Serif} 
\setmonofont{Fira Code} 
\setsansfont{Liberation Sans} 

\PolyglossiaSetup{russian}{indentfirst=true}

\newfontfamily{\cyrillicfont}{Liberation Serif}[Ligatures=TeX]
\newfontfamily{\cyrillicfontrm}{Liberation Serif}[Ligatures=TeX]
\newfontfamily{\cyrillicfonttt}{Fira Code}[Ligatures=TeX]
\newfontfamily{\cyrillicfontsf}{Liberation Sans}[Ligatures=TeX]

\usepackage{wasysym,amssymb,amsthm}

\usepackage{pgfplots}
\usetikzlibrary{arrows,positioning,shapes,calc}

\usepackage{minted}
\newminted{cpp}{frame=leftline,framerule=1.5pt,framesep=1em,breaklines=true,autogobble=true,tabsize=4}

\usetheme{default}
\usecolortheme{seagull}
\useoutertheme[subsection=false]{miniframes}
\setbeamertemplate{mini frames}[tick]

\usepackage{ragged2e}
\usepackage{xcolor}

\theoremstyle{definition}
\newtheorem{myDef}{Определение}

\usepackage{csquotes}
\usepackage[%
backend=biber,% движок
bibencoding=utf8,% кодировка bib файла
sorting=none,% настройка сортировки списка литературы
style=gost-numeric,% стиль цитирования и библиографии (по ГОСТ)
language=autobib,% получение языка из babel/polyglossia, default: autobib % если ставить autocite или auto, то цитаты в тексте с указанием страницы, получат указание страницы на языке оригинала
autolang=other,% многоязычная библиография
clearlang=true,% внутренний сброс поля language, если он совпадает с языком из babel/polyglossia
defernumbers=true,% нумерация проставляется после двух компиляций, зато позволяет выцеплять библиографию по ключевым словам и нумеровать не из большего списка
sortcites=true,% сортировать номера затекстовых ссылок при цитировании (если в квадратных скобках несколько ссылок, то отображаться будут отсортированно, а не абы как)
doi=false,% Показывать или нет ссылки на DOI
isbn=true,% Показывать или нет ISBN, ISSN, ISRN
]{biblatex}
\addbibresource{biblio.bib}

\makeatletter
\setbeamertemplate{section in head/foot}{%
    \parbox[c][0.33cm][t]{\dimexpr(\textwidth-1.3cm)/\beamer@sectionmax\relax}{%\
        \hyphenpenalty=9999\RaggedRight\fontsize{4}{4}\selectfont\insertsectionhead}}
\makeatother

\title{Лекция 7 – Иерархии классов в C++}
\institute{Кафедра прикладной математики и информатики}
\date{\today}
\begin{document}
\titlepage 
\section{Иерархии}

\begin{frame}[fragile, allowframebreaks]{Иерархии классов}
    \begin{myDef}
       \emph{Иерархия классов} -- это набор классов, упорядоченных в ориентированном графе, 
       созданной путем наследования (например, public).
    \end{myDef}
    Ранее рассмотренный абстрактный класс \mintinline{cpp}{Container} был примером простой иерархии.
    
    \vspace{2em}
    
    Мы используем иерархии классов для представления концепций, которые имеют иерархические отношения, такие как «Пожарная машина - это 
    своего рода грузовик, который является своего рода транспортным средством» и «Смайлик - это своего рода круг, который является своего 
    рода форма. В проектах распространены огромные иерархии с сотнями классов, как в глубину так и в ширину.
    
    \begin{figure}
       \begin{tikzpicture}
           \node at (0,0) (Shape) {\mintinline{cpp}{Shape}};
           \node[below left = 0.5 of Shape] (Circle){\mintinline{cpp}{Circle}};
           \node[below right = 0.5 of Shape] (Triangle){\mintinline{cpp}{Triangle}};
           \node[below left = 0.5 of Circle] (Smiley){\mintinline{cpp}{Smiley}};
           
           \path[draw,-stealth] (Shape) -- (Circle);
           \path[draw,-stealth] (Shape) -- (Triangle);
           \path[draw,-stealth] (Circle) -- (Smiley);
       \end{tikzpicture}
    \end{figure}
    Стрелки представляют отношения наследования. 
    Например, класс \mintinline{cpp}{Circle} является производным от класса \mintinline{cpp}{Shape}
    
    Чтобы представить диаграмму в коде, мы должны сначала указать класс, который определяет общие свойства всех фигур:
    \begin{cppcode}
        class Shape {
            public:
            virtual Point center() const = 0; // чистая виртуальная функция
            virtual void move(Point to) = 0;
            virtual void draw() const = 0;    // виртуальная функция отрисовки
            virtual void rotate(int angle) = 0;
            virtual ~Shape() {}	// деструктор
            // ...
        };
    \end{cppcode}
    
    Естественно, этот интерфейс является абстрактным классом: что касается представления данных,
    то ничего (кроме расположения указателя на \mintinline{cpp}{vtbl}) не является общим для каждого \mintinline{cpp}{Shape}. 
    
    Приведя это определение, мы можем написать общие функции, управляющие векторами указателей на фигуры:
    
    \begin{cppcode}
        void rotate_all(vector<Shape*>& v, int angle) // поворот всех элементов вектора на угол angle
        {
            for (auto p : v)
            p->rotate(angle);
        }
    \end{cppcode}

    
   
    
\end{frame}

\begin{frame}[fragile, allowframebreaks]{Произвольные классы}    
     Чтобы определить конкретную форму, мы должны указать, какая это форма 
    и определить ее конкретные свойства (включая ее виртуальные функции):
    \begin{cppcode}
        class Circle : public Shape {
            public:
            Circle(Point p, int rad); // конструктор
            Point center() const override
            {
                return x;
            }
            void move(Point to) override
            {
                x = to;
            }
            void draw() const override;
            void rotate(int) override {} // отличный простой алгоритм
            private:
            Point x; // центр
            int r; // радиус
        };
    \end{cppcode}

    \newpage
    Пока что в примере \mintinline{cpp}{Shape} and \mintinline{cpp}{Circle} нет ничего нового по сравнению с примером 
    \mintinline{cpp}{Container} и \mintinline{cpp}{Vector_container}, но мы можем построить дальше:
    
    {\small \begin{cppcode}
        class Smiley : public Circle { // используем Circle как родительский класс
            public:
            Smiley(Point p, int rad) : Circle{p,rad}, mouth{nullptr} { }
            ~Smiley()
            {
                delete mouth;
                for (auto p : eyes)
                delete p;
            }
            void move(Point to) override;
            void draw() const override;
            void rotate(int) override;
            void add_eye(Shape* s)
            {
                eyes.push_back(s);
            }
            void set_mouth(Shape* s);
            virtual void wink(int i); // моргнуть глазом № i 
            // ...
            private:
            vector<Shape*> eyes; // обычно 2 глаза
            Shape* mouth;
        };
    \end{cppcode}
    }

   Теперь мы можем определить \mintinline{cpp}{Smiley::draw()}, используя вызовы родительского класса \mintinline{cpp}{Circle} и других членов класса:
   {\small \begin{cppcode}
       void Smiley::draw() const
       {
           Circle::draw();
           for (auto p : eyes)
           p->draw();
           mouth->draw();
       }
   \end{cppcode} 
    }
    
    Обратите внимание на то, как \mintinline{cpp}{Smiley} следит за вектором стандартной библиотеки и удаляет их в своем деструкторе.
    Деструктор \mintinline{cpp}{Shape} виртуален, а деструктор \mintinline{cpp}{Smiley} переопределяет его. 
    Виртуальный деструктор необходим для абстрактного класса, поскольку объектом производного
    класса обычно манипулируют через интерфейс, предоставляемый его абстрактным базовым классом. 
    В частности, он может быть удален через указатель на базовый класс.
    Затем механизм вызова виртуальной функции обеспечивает вызов надлежащего деструктора. 
    Затем этот деструктор неявно вызывает деструкторы его родительского класса и членов.
    
    \vspace{1em}
    В этом упрощенном примере задачей программиста является размещение глаз и рта соответствующим образом внутри круга,
    представляющего лицо.
    Мы можем добавить поля данных, методы или и то и другое, поскольку мы определяем новый класс путем наследования.
    Это дает большую гибкость с соответствующими возможностями для путаницы и плохого дизайна.
    
\end{frame}

\section{Плюсы иерархии классов}

\begin{frame}[fragile]{Польза}
    Иерархия классов предлагает два вида преимуществ:
    \begin{itemize}
        \item \emph{Наследование интерфейса}: объект производного класса можно использовать везде, где требуется объект базового класса. 
        Таким образом, базовый класс действует как интерфейс для производного класса. Классы  \mintinline{cpp}{Container} и  
        \mintinline{cpp}{Shape} являются примерами. Такие классы часто являются абстрактными классами.
        
        \item \emph{Наследование реализации}: базовый класс предоставляет функции или данные, которые упрощают реализацию производных 
        классов. Примеры использования в \mintinline{cpp}{Smiley} конструктора \mintinline{cpp}{Circle} и 
        \mintinline{cpp}{Circle::draw ()}. Такие базовые  классы часто имеют поля-члены и конструкторы.
    \end{itemize}
\end{frame}

\begin{frame}[fragile]{Concrete vs base class}
    Конкретные классы -- особенно классы с небольшими представлениями -- очень похожи на встроенные типы: мы определяем их как локальные 
    переменные, обращаемся к ним по их именам, копируем их и т.д. Классы в иерархиях классов различны: мы склонны размещать их в куче 
    (динамической памяти) с использованием ключевого слова \mintinline{cpp}{new}, и мы получаем к ним доступ через указатели или ссылки. 
    
    \begin{block}{note}
        оба типа классов в русском языке тяготеет к переводу "базовый класс".
    \end{block}
    
\end{frame}

\begin{frame}[fragile,allowframebreaks]{Пример чтения}
    Рассмотрим функцию, которая считывает данные, описывающие фигуры, из входного потока и создает соответствующие объекты \mintinline{cpp}{Shape}:
    {\small \begin{cppcode}
        enum class Kind { circle, triangle , smiley };
        Shape* read_shape(istream& is) // читает описание Shape из istream is
        {
            // ... чтение заголовка, чтобы определить Kind k ...
            switch (k) {
                case Kind::circle:
                // read circle data {Point,int} into p and r
                return new Circle{p,r};
                case Kind::triangle:
                // read triangle data {Point,Point,Point} into p1, p2, and p3
                return new Triangle{p1,p2,p3};
                case Kind::smiley:
                // read smiley data {Point,int,Shape,Shape,Shape} into p, r, e1, e2, and m
                Smiley* ps = new Smiley{p,r};
                ps->add_eye(e1);
                ps->add_eye(e2);
                ps->set_mouth(m);
                return ps;
            }
        }
    \end{cppcode}
    }

    \newpage
    \small
    Программа может использовать такое чтение формы следующим образом:
    \begin{cppcode}
        void user()
        {
            std::vector<Shape*> v;
            while (cin)
            v.push_back(read_shape(cin));
            draw_all(v); //вызов draw() для каждого элемента
            rotate_all(v,45); //вызов rotate(45) для каждого элемента
            for (auto p : v) // не забываем удалять данные из вектора
                delete p;
        }
    \end{cppcode}
    Очевидно, что пример упрощен -- особенно в отношении обработки ошибок - но он наглядно демонстрирует, что \mintinline{cpp}{user()}  
    абсолютно не представляет, с какими формами он манипулирует. Код \mintinline{cpp}{user()} может быть скомпилирован один раз, а затем 
    использован для новых фигур, добавленных в программу. Обратите внимание, что нет указателей на фигуры вне \mintinline{cpp}{user()} , 
    поэтому \mintinline{cpp}{user()} отвечает за их освобождение. Это делается с помощью оператора удаления и критически зависит от 
    виртуального деструктора \mintinline{cpp}{Shape}. Поскольку этот деструктор является виртуальным, \mintinline{cpp}{delete} вызывает 
    деструктор для самого производного класса. Это очень важно, поскольку производный класс может получить все виды ресурсов (например, 
    файловые дескрипторы, блокировки и выходные потоки), которые должны быть освобождены. В этом случае Смайлик удаляет свои глаза и 
    предметы рта. После этого он вызывает деструктор Круга. Объекты конструируются «снизу вверх» конструкторами, а 
    «разрушаются сверху» деструкторами.
\end{frame}

\section{Навигация в иерархии}
\begin{frame}[fragile]{Определение класса наследника}
    Функция \mintinline{cpp}{read_shape()} возвращает \mintinline{cpp}{Shape*}, чтобы мы могли обрабатывать все объекта типа \mintinline{cpp}{Shape} одинаково. Однако что мы можем сделать, если мы хотим использовать функцию-член, которая предоставляется только определенным производным классом, таким как \mintinline{cpp}{Smiley::wink()}? Мы можем спросить: «Эта форма -- своего рода смайлик?», Используя оператор \mintinline{cpp}{dynamic_cast}:
    \begin{cppcode}
        Shape* ps {read_shape(cin)};
        if (Smiley* p = dynamic_cast<Smiley*>(ps)) { // ... указатель на Smiley? ...
            // ... Smiley; используем 
        }
        else {
            // ... не Smiley, пробуем что-то еще ...
        }
    \end{cppcode}

    Если во время выполнения объект, на который указывает аргумент \mintinline{cpp}{dynamic_cast} (\mintinline{cpp}{ps}), 
    не относится к ожидаемому типу (здесь \mintinline{cpp}{Smiley}) или к классу, производному от ожидаемого типа, 
    \mintinline{cpp}{dynamic_cast} возвращает \mintinline{cpp}{nullptr}.
\end{frame}

\begin{frame}[fragile]{Исключение при \mintinline{cpp}{dynamic_cast}}
    Мы используем \mintinline{cpp}{dynamic_cast} для типа указателя, когда указатель на объект другого производного класса является допустимым аргументом. 
    Затем мы проверяем, является ли результат нулевым. Этот тест часто удобно помещать в инициализацию переменной в условии.
    \vspace{1em}
    
    Когда другой тип неприемлем, мы можем просто провести \mintinline{cpp}{dynamic_cast} на ссылочный тип. 
    Если объект не относится к ожидаемому типу, \mintinline{cpp}{dynamic_cast} генерирует исключение \mintinline{cpp}{bad_cast}:
    
    \begin{cppcode}
        Shape* ps {read_shape(cin)};
        Smiley& r {dynamic_cast<Smiley&>(*ps)}; // где-то позже обработаем исключение std::bad_cast
    \end{cppcode}

    
\end{frame}

\begin{frame}[fragile]{\mintinline{cpp}{dynamic_cast}}
    
    Код чище, когда \mintinline{cpp}{dynamic_cast} используется со сдержанностью. Если мы можем избежать использования информации о 
    типе, мы можем написать более простой и эффективный код, но иногда информация о типе теряется и должна быть восстановлена. Обычно 
    это происходит, когда мы передаем объект в какую-либо систему, которая принимает интерфейс, заданный базовым классом. Когда эта 
    система позже передаст объект нам. Нам, возможно, придется восстановить исходный тип. Операции, аналогичные 
    \mintinline{cpp}{dynamic_cast}, известны как "является своего рода", "является экземпляром".
\end{frame}

\section{Избегаем утечки ресурсов}
\begin{frame}[fragile]{Ошибки в реализации}
    Опытные программисты заметят, что мы оставили открытыми три возможности для ошибок:
    \begin{itemize}
        \item Разработчик \mintinline{cpp}{Smiley} может не справиться с удалением указателя  \mintinline{cpp}{mouth}.
        \item Пользователь \mintinline{cpp}{read_shape()} может не справиться с удалением возвращаемого указателя.
        \item Владелец контейнера содержащего набор \mintinline{cpp}{Shape*} может не справиться с удалением объектов, на которые указывает указатель.
    \end{itemize}

    В этом смысле указатели на объекты, размещенные в динамической памяти, опасны: «простой старый указатель» не должен использоваться для представления владения ресурсом. Например:
    
    \begin{cppcode}
        void user(int x)
        {
            Shape* p = new Circle{Point{0,0},10};
            // ...
            if (x<0) throw Bad_x{}; // возможная утечка
            if (x==0) return; // возможная утечка
            // ...
            delete p;
        }
    \end{cppcode}

    Это утечкой памяти, если \mintinline{cpp}{x} не является положительным. 
    Присваивание результата оператора \mintinline{cpp}{new} «голому указателю» вызывает проблемы.
\end{frame}

\begin{frame}[fragile,allowframebreaks]{Избегаем проблем с памятью}
    Одним простым решением таких проблем является использование \mintinline{cpp}{unique_ptr} из стандартной библиотеки, а не «голый указатель», когда требуется удаление:
    \begin{cppcode}
        class Smiley : public Circle {
            // ...
            private:
            vector<unique_ptr<Shape>> eyes; // обычно два глаза
            unique_ptr<Shape> mouth;
        };
    \end{cppcode}
    Это пример простого, общего и эффективного метода управления ресурсами.
    \vspace{1em}
    
    Как приятный побочный эффект этого изменения, нам больше не нужно определять деструктор для \mintinline{cpp}{Smiley}. Компилятор неявно сгенерирует тот, который выполняет требуемое уничтожение \mintinline{cpp}{unique_ptr} в векторе. Код, использующий \mintinline{cpp}{unique_ptr}, будет так же эффективен, как и код, использующий голые указатели.
    
    \newpage
    
    \small
    Теперь рассмотрим использование \mintinline{cpp}{read_shape()}:
    \begin{cppcode}
        unique_ptr<Shape> read_shape(istream& is) // читает описание Shape из istream is
        {
            // ... чтение заголовка, чтобы определить Kind k ...
            switch (k) {
                case Kind::circle:
                // read circle data {Point,int} into p and r
                return unique_ptr<Shape>{new Circle{p,r}}; 
                // ...
            }
        }
    \end{cppcode}

    \begin{cppcode}
        void user()
        {
            vector<unique_ptr<Shape>> v;
            while (cin)
            v.push_back(read_shape(cin));
            draw_all(v); //вызываем draw() для каждого элемента
            rotate_all(v,45); //вызываем rotate(45) для каждого элемента
        } // все Shape неявно разрушены
    \end{cppcode}
    Теперь каждый объект принадлежит \mintinline{cpp}{unique_ptr}, который будет удалять объект,
    когда он больше не нужен, то есть когда его \mintinline{cpp}{unique_ptr} выходит из области видимости.
    
    Чтобы работала версия \mintinline{cpp}{unique_ptr} \mintinline{cpp}{user()}, нам нужны версии \mintinline{cpp}{draw_all()} и \mintinline{cpp}{rotate_all()}, которые принимают\mintinline{cpp}{vector<unique_ptr<Shape>>}.
\end{frame}

\section{Рекомендации}

\begin{frame}[fragile,allowframebreaks]{Рекомедации}
    \begin{enumerate}
        \item Выражай идеи прямо в коде.
        \item Конкретный тип (concrete type) -- самый простой вид класса. Где это применимо, предпочитайте конкретный тип более сложным классам и простым структурам данных.
        \item Используйте конкретные классы для представления простых концепций.
        \item Предпочитайте конкретные классы иерархиям классов для компонентов, критичных к производительности.
        \item Определяйте конструкторы для инициализации объектов.
        \item Объявляйте функцию членом, только если ей нужен прямой доступ к представлению данных класса.
        \item Определяйте оператороы в первую очередь для имитации обычного использования.
        \item Используйте свободные функции для симметричных операторов.
        \item Объявляйте функцию-член, которая не изменяет состояние своего объекта \mintinline{cpp}{const}.
        \item Если конструктор получает ресурс, его классу нужен деструктор для освобождения ресурса.
        \item Избегайте «голых» операций \mintinline{cpp}{new} и \mintinline{cpp}{delete}.
        \item Используйте дескрипторы ресурсов и RAII для управления ресурсами.
        \item Если класс является контейнером, создайте ему конструктор списка инициализаторов.
        \item Используйте абстрактные классы в качестве интерфейсов, когда необходимо полное разделение интерфейса и реализации.
        \item Получайте доступ к полиморфным объектам через указатели и ссылки.
        \item Абстрактный класс обычно не нуждается в конструкторе.
        \item Используйте иерархии классов для представления концепций с внутренней иерархической структурой.
        \item Класс с виртуальной функцией должен иметь виртуальный деструктор.
        \item Используйте \mintinline{cpp}{override}, чтобы указать переопределение явным в больших иерархиях классов.
        \item При разработке иерархии классов следует различать наследование реализации и наследование интерфейса.
        \item Используйте \mintinline{cpp}{dynamic_cast}, где навигация по иерархии классов неизбежна.
        \item Используйте \mintinline{cpp}{dynamic_cast} для ссылочного типа, когда неудача получения требуемого класса является ошибкой.
        \item Используйте \mintinline{cpp}{dynamic_cast} для типа указателя, когда неудача получения требуемого класса является допустимой альтернативой.
        \item Используйте \mintinline{cpp}{unique_ptr} или \mintinline{cpp}{shared_ptr}, чтобы не забыть удалить объекты, созданные с помощью \mintinline{cpp}{new}.
    \end{enumerate}
\end{frame}




\end{document}