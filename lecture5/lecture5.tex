% !TeX encoding = UTF-8
% !TeX spellcheck = ru_RU
% !TeX root = lecture5.tex

\documentclass[
    9pt,
    hyperref={pdfencoding=unicode}
    ]{beamer}

\usepackage{polyglossia}   %% загружает пакет многоязыковой вёрстки

\setmainlanguage{russian}  %% устанавливает главный язык документа
\setmainfont{Liberation Serif}
\setromanfont{Liberation Serif} 
\setmonofont{Fira Code} 
\setsansfont{Liberation Sans} 

\PolyglossiaSetup{russian}{indentfirst=true}

\newfontfamily{\cyrillicfont}{Liberation Serif}[Ligatures=TeX]
\newfontfamily{\cyrillicfontrm}{Liberation Serif}[Ligatures=TeX]
\newfontfamily{\cyrillicfonttt}{Fira Code}[Ligatures=TeX]
\newfontfamily{\cyrillicfontsf}{Liberation Sans}[Ligatures=TeX]

\usepackage{wasysym,amssymb,amsthm}

\usepackage{pgfplots}
\usetikzlibrary{arrows,positioning,shapes}

\usepackage{minted}
\newminted{cpp}{frame=leftline,framerule=1.5pt,framesep=1em,breaklines=true,autogobble=true,tabsize=4}

\usetheme{default}
\usecolortheme{seagull}
\useoutertheme[subsection=false]{miniframes}
\setbeamertemplate{mini frames}[tick]

\usepackage{ragged2e}
\usepackage{xcolor}

\theoremstyle{definition}
\newtheorem{myDef}{Определение}

\usepackage{csquotes}
\usepackage[%
backend=biber,% движок
bibencoding=utf8,% кодировка bib файла
sorting=none,% настройка сортировки списка литературы
style=gost-numeric,% стиль цитирования и библиографии (по ГОСТ)
language=autobib,% получение языка из babel/polyglossia, default: autobib % если ставить autocite или auto, то цитаты в тексте с указанием страницы, получат указание страницы на языке оригинала
autolang=other,% многоязычная библиография
clearlang=true,% внутренний сброс поля language, если он совпадает с языком из babel/polyglossia
defernumbers=true,% нумерация проставляется после двух компиляций, зато позволяет выцеплять библиографию по ключевым словам и нумеровать не из большего списка
sortcites=true,% сортировать номера затекстовых ссылок при цитировании (если в квадратных скобках несколько ссылок, то отображаться будут отсортированно, а не абы как)
doi=false,% Показывать или нет ссылки на DOI
isbn=true,% Показывать или нет ISBN, ISSN, ISRN
]{biblatex}
\addbibresource{biblio.bib}

\makeatletter
\setbeamertemplate{section in head/foot}{%
    \parbox[c][0.33cm][t]{\dimexpr(\textwidth-1.3cm)/\beamer@sectionmax\relax}{%\
        \hyphenpenalty=9999\RaggedRight\fontsize{4}{4}\selectfont\insertsectionhead}}
\makeatother

\title{Лекция 5 – Классы в языке C++}
\institute{Кафедра прикладной математики и информатики}
\date{\today}
\begin{document}
\titlepage 
\section{Введение}

\begin{frame}[fragile]{Определение класса}
    Данные, объявленные отдельно от операций над ними, предоставляют больше возможностей по использованию таких данных. Тем не менее более тесная связь между представлением данных и
    операцями над ними необходима для того, чтобы определенный пользователем тип имел все свойства, ожидаемые от типа, представляющего объект реального мира.

    \vfill
    
    Часто нам надо сохранить представление недоступным для пользователей,
    чтобы упростить использование, гарантировать последовательное использование
    данных, сохраняя при этом возможноть улучшить представление. Для этого мы должны
    различать публично (\mintinline{cpp}{public}) доступные данные (который будет использоваться всеми) 
    и скрытую (\mintinline{cpp}{private}) реализацию пользовательского типа (которая
    имеет доступ к недоступным  в противном случае данным). 
    
    \vfill
    
    Языковой механизм для этого называется \emph{классом}.     
\end{frame}

\begin{frame}[fragile]
    
    Например:
    
    \begin{cppcode}
        class Vector {
          public:
            // конструктор класса Vector
            Vector(int s) :elem{new double[s]}, sz{s} { } 
            
            // деструктор класса Vector
            ~Vector(){ delete[] elem; }
            
            // доступ к элементу массива
            double& operator[](int i) { return elem[i]; } 
            
            // размер вектора
            int size() { return sz; }
            
          private:
            double* elem; // указатель на элементы вектора
            int sz; // количество элементов
        };
    \end{cppcode}    
      
         
     \vfill
      
    Имея такое определение, мы можем объявить переменную типа \mintinline{cpp}{Vector}:
    \begin{cppcode}
        Vector v(6);
    \end{cppcode}   
\end{frame}

\begin{frame}{Представление объектов в памяти}
    \begin{figure}
        \begin{tikzpicture}[align=center,font=\ttfamily]
        \node (Vec) at (0,0) {\textbf{Vector:}};
        \node[draw,rectangle,below = 0 of Vec,minimum height=5mm, minimum width = 10mm] (VecElem) {};
        \node[draw,rectangle,below = 0 of VecElem,minimum height=5mm, minimum width = 10mm] (VecSZ) {6};
        
        \node[left = 0 of VecElem] (elem) {\textbf{elem:}};
        \node[left = 0 of VecSZ] (sz) {\textbf{sz:}};
        
        \node[draw,rectangle,below right = 1mm and 20mm of VecElem,minimum height=5mm, minimum width = 10mm] (v0) {};
        \node[draw,rectangle,right = 0 of v0,minimum height=5mm, minimum width = 10mm] (v1) {};
        \node[draw,rectangle,right = 0 of v1,minimum height=5mm, minimum width = 10mm] (v2) {};
        \node[draw,rectangle,right = 0 of v2,minimum height=5mm, minimum width = 10mm] (v3) {};
        \node[draw,rectangle,right = 0 of v3,minimum height=5mm, minimum width = 10mm] (v4) {};
        \node[draw,rectangle,right = 0 of v4,minimum height=5mm, minimum width = 10mm] (v5) {};
        
        \node[above =1mm of v0] (n0) {0:};
        \node[above =1mm of v1] (n1) {1:};
        \node[above =1mm of v2] (n2) {2:};
        \node[above =1mm of v3] (n3) {3:};
        \node[above =1mm of v4] (n4) {4:};
        \node[above =1mm of v5] (n5) {5:};
  
        \path[draw,-stealth] (VecElem.center) -- (v0.west);
        \end{tikzpicture}
    \end{figure}
    
    \vfill
    
    Объект вектора представляет собой обертку, содержащую указатель на элементы (\mintinline{cpp}{elem}) и количество элементов (\mintinline{cpp}{sz}).
    Количество элементов (6 в нашем случае) может варьироватся от одного объекта типа \mintinline{cpp}{Vector} к другому объекту типа \mintinline{cpp}{Vector}.
    
    \vfill
    
    Однако, сам объект типа \mintinline{cpp}{Vector} имеет фиксированный размер.
    
    Это базовая техника для обработки переменного объема информации в C++:
    обработчик фиксированного размера ссылается на изменяющееся количество информации 
    "где-то еще" (например аллоцированную через оператор \mintinline{cpp}{new}). 
\end{frame}


\begin{frame}[fragile]{Использование публичного интерфейса}
    Ниже представлено использование приватных полей класса \mintinline{cpp}{Vector} через публичный члены класса \mintinline{cpp}{Vector()},  \mintinline{cpp}{operator[]()} и 
    \mintinline{cpp}{size()}.
    
    \vfill
    
    \begin{cppcode}
        double read_and_sum(int s)
        {
            Vector v(s); // создаем вектор s элементов
            for (int i=0; i!=v.size(); ++i)
                cin>>v[i]; // читаем содержимое из консоли
            double sum = 0;
            for (int i=0; i!=v.size(); ++i)
                sum+=v[i]; // считаем сумму элементов
            return sum;
        }
    \end{cppcode}

"Функция" член с тем же самым именем что и класс называется \emph{конструктором}, которая используется для создания объектов класса.

\vfill

\mintinline{cpp}{Vector(int)} определяет, каким образом конструируется объект типа \mintinline{cpp}{Vector}. В частности требуется аргумент типа int для того, чтобы создать объект. Это целое число используется как количество элементов вектора.
\end{frame}

\begin{frame}    
    Конструктор инициализиурет члены класса \mintinline{cpp}{Vector} 
    при помощи списка инициализации:
    \mintinline{cpp}{:elem{new double[s]}, sz{s}}
    
    \vfill
    
    Таким образом мы инициализируем переменную член \mintinline{cpp}{elem} указателем на \mintinline{cpp}{s} элементов типа \mintinline{cpp}{double} полученных из свободного пулла памяти операционной системы.
    Затем мы инициализируем \mintinline{cpp}{sz} числом \mintinline{cpp}{s}.
    
    \vfill
    
    Доступ к элементам предоставляется через квадратные скобки, аналогично массиву встроенному в язык C++. Определение такого доступа соответствует \mintinline{cpp}{operator[]}, которое предоставлет доступ как на чтение, так и на запись значений вектора.
    
    \vfill
    
    Функция \mintinline{cpp}{size()} возвращает количество элементов в векторе.
    
    \vfill
    
    "Функция" \mintinline{cpp}{~Vector()} называется \emph{деструктором} класса. 
    Она предназначена для возвращения аллоцированной памяти обратно в пулл операционной системы. Деструктор вызывается неявно при вытеснении переменной \mintinline{cpp}{v} со стека.
    
\end{frame}

\section{Базовый класс}
\begin{frame}[fragile]{Характеристика базового класса}
    Основная идея базового класса состоит в том, что его поведение "такое же как и у встроенных типов данных". Например тип комплексных чисел ведет себя так же как и встроенный тип \mintinline{cpp}{int}, за исключением, конечно, их собственной семантики и набора операций. Также \mintinline{cpp}{vector} или \mintinline{cpp}{string} похожы на встроенные массивы, только с лучшим поведением.
    
    \vfill
    
    Основная характеристика базового класса состоит в том, что представление данных (representation) класса есть часть его определения. Во множестве случаев, таких как \mintinline{cpp}{vector} представление данных в виде одного или нескольких указателей на данные хранящиеся "где-то еще". Но это представление данных существует в каждом объекте базового класса. Это позволяет создавать реализации эффективные по скорости и используемой памяти. В частности это дает:
    \begin{itemize}
        \item размещать объекты базового класса на стеке, в статической памяти и других объектах;
        \item обращаться к объету напрямую (не просто через указатели и ссылки);
        \item инициализировать объекты сразу и полностью (например используя конструкторы);
        \item копировать и перемещать объекты.
    \end{itemize}
    
\end{frame}

\begin{frame}[fragile]{Пример арифметического типа}
    \footnotesize
    \begin{cppcode}
        class complex {
            double re, im; // representation: два числа double
            public:
            complex(double r, double i) :re{r}, im{i} {} // конструктор двух аргументов
            complex(double r) :re{r}, im{0} {} // конструктор одного аргумента
            complex() :re{0}, im{0} {} // конструктор по умолчанию complex: {0,0}
            double real() const { return re; }
            void real(double d) { re=d; }
            double imag() const { return im; }
            void imag(double d) { im=d; }
            complex& operator+=(complex z)
            {
                re+=z.re; // сумма по определению комплексных чисел
                im+=z.im;
                return *this; // возвращение результата
            }
            complex& operator-=(complex z)
            {
                re-=z.re;
                im-=z.im;
                return *this;
            }
            complex& operator*=(complex); // определено вне класса
            complex& operator/=(complex); // определено вне класса
        };
    \end{cppcode}
\end{frame}

\begin{frame}
    Приведенная ранее реализация комплексных чисел представляет собой упрощенную версию класса  \mintinline{cpp}{complex} из стандартной библиотеки.
    
    \vfill
    
    Определение класса содержит только те операции, которым необходим доступ к 
    представлению данных класса. Представление скрыто за приватным модификатором 
    доступа. Само по себе представление данных простое и понятное: два числа типа 
    \mintinline{cpp}{double} представляющие действительную и мнимую часть 
    комплексного числа.
    
    \vfill
    
    По умолчанию функции члены, определенные внутри класса, имеют модификатор 
    \mintinline{cpp}{inline}, который объявлен неявно. Этот модификатор позволяет 
    компилятору избавиться от функционального вызова и напрямую генерировать байт код 
    для соответствующей операции.
    
    \vfill
    
    Конструктор, который может быть вызван без аргументов называется 
    \emph{конструктором по умолчанию}. Таким образом \mintinline{cpp}{complex()} -- 
    конструктор по умолчанию класса комплексных чисел. При объявлении конструктора по 
    умолчанию избегается ситуация создания не инициализированных переменных данного 
    типа.
    
\end{frame}

\begin{frame}[fragile]{Const}
    Квалификатор \mintinline{cpp}{const}, примененный к функциям возвращающим действительную и мнимую часть комплексного числа, сообщает о том, что функция не изменяет состояние объекта, для которого она вызвана.
    
    \vfill
    
    Константная функция член может быть вызвана для константных и неконстантных объектов, но неконстантная функция член может быть вызвана только для неконстантных объектов. Например:
    
    \vfill
    
    \begin{cppcode}
        complex z = {1,0};
        const complex cz {1,3};
        z =cz; //OK: assigning to a non-const var iable
        cz = z; // error : complex::operator=() is a non-const member function
        double x = z.real(); // OK: complex::real() is a const member function
    \end{cppcode}
\end{frame}


\begin{frame}[fragile]{Операторы}
    Множество полезных операторов не требуют прямого доступа к представлению данных класса \mintinline{cpp}{complex}, таким образом их можно определить вне класса:
    
    \vfill
    
    \begin{cppcode}
        complex operator+(complex a, complex b) 
        { return a+=b; }
        complex operator-(complex a, complex b) 
        { return a-=b; }
        complex operator-(complex a) // унарный минус
        { return {-a.real(), -a.imag()}; } 
        complex operator*(complex a, complex b) 
        { return a*=b; }
        complex operator/(complex a, complex b) 
        { return a/=b; }
    \end{cppcode}

    \vfill
    
    Здесь использован факт, что аргументы передаваемые по значению копируются, 
    таким образом изменение локальной переменной \mintinline{cpp}{a} не влечет изменения оригинальной переменной, с которой вызвана функция.

\end{frame}
  
    
\begin{frame}[fragile]
   Определение операторов \mintinline{cpp}{==} и \mintinline{cpp}{!=} весьма прямолинейно:
   \begin{cppcode}
       bool operator==(complex a, complex b) // equal
       {
           return a.real()==b.real() && a.imag()==b.imag();
       }
       bool operator!=(complex a, complex b) // not equal
       {
           return !(a==b);
       }
   \end{cppcode} 
\end{frame}

\begin{frame}[fragile]
    Использование класса может быть таким:
    \begin{cppcode}
        void f(complex z)
        {
            complex a {2.3}; // конструируем {2.3,0.0} из 2.3
            complex b {1/a};
            complex c {a+z*complex{1,2.3}};
            // ...
            if (c != b)
                c = -(b/a)+2*b;
            // ...
        }
        
    \end{cppcode} 
\end{frame}

\begin{frame}[fragile]{Перегрузка операторов}
    Компилятор преобразует операторы с участием комплексных чисел в корректные функциональные вызовы. Например, \mintinline{cpp}{c!=b} означает \mintinline{cpp}{operator!=(c,b)} и \mintinline{cpp}{1/a} означает \mintinline{cpp}{operator/(complex{1},a)}
    
    \vfill
    
    Определенные пользователем операторы ("перегруженные операторы") должны быть реализованы с осторожностью и умом. Синтакс данных операторов фиксирован языком, так что программист не может определить унарный оператор \mintinline{cpp}{/}. Также, невозможно переопределить значение оператора для встроенного типа, тоесть нельзя переопределить \mintinline{cpp}{+} для вычитания двух значений типа \mintinline{cpp}{int}.
    
\end{frame}

\section{Рекомендации}
\begin{frame}[t,allowframebreaks]{Советы}
    Изучай:    \href{https://github.com/isocpp/CppCoreGuidelines/blob/master/CppCoreGuidelines.md}{https://github.com/isocpp/CppCoreGuidelines/blob/master/CppCoreGuidelines.md}
    
    \begin{enumerate}
        \item Выражай идеи непосредственно в коде.
        \item Базовый класс -- простейшая форма класса. Когда возможно используй базовый класс вместо более сложных классов и вместо простых структур данных.
        \item Используй базовые классы для представления простейших концептов.
        \item Используй базовые классы вместо иерархий классов для критических компонентов требующих высокой производительности.
        \item Определяй конструкторы для инициализации объекта.
        \item Создавай функцию член только тогда, когда ей требуется непосредственный доступ к представлению данных класса.
        \item Определяй операторы преимущественно для имитации их стандартного использования.
        \item Используй свободные функции для определения симметричных операторов.
        \item Объявляй функции члены константными, если они не изменяют состояния объекта.
        \item Если конструктор получает ресурс, этот класс нуждается в деструкторе, который этот ресурс освобождает.
        \item Избегай "голых" операторов \mintinline{cpp}{new} и \mintinline{cpp}{delete}.
        \item Используй обертки для того чтобы управлять ресурсом.
    \end{enumerate}
\end{frame}

% https://ravesli.com/urok-85-dinamicheskoe-vydelenie-pamyati-operatory-new-i-delete/
% https://code-live.ru/post/cpp-array-tutorial-part-2/#_2

%\begin{frame}[t,allowframebreaks]
%    \frametitle{Литературные источники}
%    
%    \nocite{*}
%    \printbibliography
%\end{frame}


\end{document}