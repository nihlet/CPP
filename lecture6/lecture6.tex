% !TeX encoding = UTF-8
% !TeX spellcheck = ru_RU
% !TeX root = lecture6.tex

\documentclass[
    8pt,
    hyperref={pdfencoding=unicode}
    ]{beamer}

\usepackage{polyglossia}   %% загружает пакет многоязыковой вёрстки

\setmainlanguage{russian}  %% устанавливает главный язык документа
\setmainfont{Liberation Serif}
\setromanfont{Liberation Serif} 
\setmonofont{Fira Code} 
\setsansfont{Liberation Sans} 

\PolyglossiaSetup{russian}{indentfirst=true}

\newfontfamily{\cyrillicfont}{Liberation Serif}[Ligatures=TeX]
\newfontfamily{\cyrillicfontrm}{Liberation Serif}[Ligatures=TeX]
\newfontfamily{\cyrillicfonttt}{Fira Code}[Ligatures=TeX]
\newfontfamily{\cyrillicfontsf}{Liberation Sans}[Ligatures=TeX]

\usepackage{wasysym,amssymb,amsthm}

\usepackage{pgfplots}
\usetikzlibrary{arrows,positioning,shapes,calc}

\usepackage{minted}
\newminted{cpp}{frame=leftline,framerule=1.5pt,framesep=1em,breaklines=true,autogobble=true,tabsize=4}

\usetheme{default}
\usecolortheme{seagull}
\useoutertheme[subsection=false]{miniframes}
\setbeamertemplate{mini frames}[tick]

\usepackage{ragged2e}
\usepackage{xcolor}

\theoremstyle{definition}
\newtheorem{myDef}{Определение}

\usepackage{csquotes}
\usepackage[%
backend=biber,% движок
bibencoding=utf8,% кодировка bib файла
sorting=none,% настройка сортировки списка литературы
style=gost-numeric,% стиль цитирования и библиографии (по ГОСТ)
language=autobib,% получение языка из babel/polyglossia, default: autobib % если ставить autocite или auto, то цитаты в тексте с указанием страницы, получат указание страницы на языке оригинала
autolang=other,% многоязычная библиография
clearlang=true,% внутренний сброс поля language, если он совпадает с языком из babel/polyglossia
defernumbers=true,% нумерация проставляется после двух компиляций, зато позволяет выцеплять библиографию по ключевым словам и нумеровать не из большего списка
sortcites=true,% сортировать номера затекстовых ссылок при цитировании (если в квадратных скобках несколько ссылок, то отображаться будут отсортированно, а не абы как)
doi=false,% Показывать или нет ссылки на DOI
isbn=true,% Показывать или нет ISBN, ISSN, ISRN
]{biblatex}
\addbibresource{biblio.bib}

\makeatletter
\setbeamertemplate{section in head/foot}{%
    \parbox[c][0.33cm][t]{\dimexpr(\textwidth-1.3cm)/\beamer@sectionmax\relax}{%\
        \hyphenpenalty=9999\RaggedRight\fontsize{4}{4}\selectfont\insertsectionhead}}
\makeatother

\title{Лекция 6 – Классы в языке C++}
\institute{Кафедра прикладной математики и информатики}
\date{\today}
\begin{document}
\titlepage 
\section{Контейнер}

\begin{frame}[fragile, allowframebreaks]{Класс контейнер}
    \begin{myDef}
        \emph{Контейнером} называется объект содержащий коллекцию из нескольких элементов.
    \end{myDef}
    Мы называем класс \mintinline{cpp}{Vector} контейнером, потому что он предназначен для хранения коллекции нескольких элементов.

    
    Пара функций конструктор и деструктор позволяет создавать коллекции неизвестного на 
    момент исполнения размера.
    
    Техника получения ресурсов в конструкторе и освобождения их в деструкторе, 
    известная как Resource Acquisition Is Initialization или RAII, 
    позволяет нам исключить «голые» операции \mintinline{cpp}{new}, то есть избежать
    выделения памяти в общем коде и сохранить такие выделяния скрытыми внутри реализации
    хороших поведенческих абстракций.
    Точно так же следует избегать «голой» операции \mintinline{cpp}{delete}. Скрытие абстракцией \mintinline{cpp}{new} и \mintinline{cpp}{delete} делает код намного менее подверженным ошибкам и намного легче избежать утечек ресурсов.
    
    \newpage
    В предыдущих лекция мы уже видели пример класса контейнера:
    
        \begin{cppcode}
        class Vector {
            public:
            // конструктор класса Vector
            Vector(int s) :elem{new double[s]}, sz{s} { } 
            
            // деструктор класса Vector
            ~Vector(){ delete[] elem; }
            
            // доступ к элементу массива
            double& operator[](int i) { return elem[i]; } 
            
            // размер вектора
            int size() { return sz; }
            
            private:
            double* elem; // указатель на элементы вектора
            int sz; // количество элементов
        };
    \end{cppcode}    
\end{frame}



\begin{frame}[fragile]{Инициализация данных в контейнере}
    Контейнер существует для хранения элементов, поэтому, очевидно, нам нужны удобные способы доставки элементов в контейнер.
    Мы можем создать вектор с соответствующим количеством элементов, а затем назначить их, но обычно другие способы более элегантны.
    
    Например:
    \begin{itemize}
        \item список инициализации
        \item \mintinline{cpp}{push_back()}
    \end{itemize}

    \begin{cppcode}
        class Vector {
            public:
            Vector(std::initializer_list<double>); // инициализация списком double
            // ...
            void push_back(double); // добавляет новый элемент в конец
            // ...
        };
    \end{cppcode}
\end{frame}


\begin{frame}[fragile]{Список инициализации}
   \mintinline{cpp}{std::initalizer_list}, используемый для определения конструктора initializer-list, является типом стандартной библиотеки, известным компилятору: когда мы используем {} как список, такой как {1,2,3,4}, компилятор создаст объект типа \mintinline{cpp}{std::initalizer_list} для передачи программе. Итак, мы можем написать:
   \begin{cppcode}
       Vector v1 = {1, 2, 3, 4, 5}; // v1 имеет 5 элементов
       Vector v2 = {1.2, 3.4, 6, 8}; // v2 имеет 4 элемента
   \end{cppcode}
    
    \vfill
    Конструктор списка инициализации вектора может быть определен так:
    
    \begin{cppcode}
        Vector::Vector(std::initializer_list<double> lst) // инициализация списком
        :elem{new double[lst.siz e()]}, sz{static_cast<int>(lst.size())}
        {
            copy(lst.begin(),lst.end(),elem); // копирование из списка в elem
        }
    \end{cppcode}
    
\end{frame}

\begin{frame}[fragile]{\mintinline{cpp}{push_back()}}
    Функция \mintinline{cpp}{push_back()} полезна для ввода произвольного числа элементов. Например:
    
    \begin{cppcode}
        Vector read(istream& is)
        {
            Vector v;
            for (double d; is>>d; ) // читаем числа с двойной точностью во временную переменную d
            v.push_back(d); // add d to v
            return v;
        }
    \end{cppcode}
    \vfill
    Цикл ввода завершается из-за конца файла или ошибки форматирования.
    До того, как это случится, каждое прочитанное число добавляется в \mintinline{cpp}{Vector}, а именно 
    добавляется в конец. Размер вектора \mintinline{cpp}{v} равняется количеству прочитанных элементов.    
\end{frame}

\section{Абстрактные типы}
\begin{frame}[fragile]{Абстрактные типы}
    Такие типы, как \mintinline{cpp}{complex} и \mintinline{cpp}{Vector}, называются \emph{базовыми} типами,
    потому что их представление является частью их определения. В этом они похожи на встроенные типы. 
    Напротив, \emph{абстрактный} тип -- это тип, который полностью изолирует пользователя от деталей реализации.
    Для этого мы отделяем интерфейс от представления и оставляем подлинные локальные переменные. 
    Поскольку мы ничего не знаем о представлении абстрактного типа (даже о его размере), мы должны 
    разместить объекты в свободном хранилище и получить к ним доступ через ссылки или указатели.
    
    
\end{frame}

\begin{frame}[fragile,allowframebreaks]{Абстрактный тип \mintinline{cpp}{Container}}
    Сначала мы определим интерфейс класса \mintinline{cpp}{Container}, который мы разработаем 
    как более абстрактную версию нашего класса \mintinline{cpp}{Vector}:
    \begin{cppcode}
        class Container {
            public:
            virtual double& operator[](int) = 0; // чистая виртуальная функция
            virtual int size() const = 0; // константная функция член
            virtual ~Container() {} // деструктор
        };
    \end{cppcode}

    Этот класс является чистым интерфейсом для определенных контейнеров, определенных позже.
    Слово \mintinline{cpp}{vitrual} означает «может быть позже переопределено в классе, производном от него». 
    Неудивительно, что функция, объявленная виртуальной, называется виртуальной функцией.
    Класс, производный от \mintinline{cpp}{Container}, обеспечивает реализацию интерфейса \mintinline{cpp}{Container}.
    
    \vspace{4mm}
    
    Синтаксис  \mintinline{cpp}{= 0} говорит, что функция чисто виртуальная; то есть некоторый класс, производный от \mintinline{cpp}{Container}, должен определять функцию.
    
    \newpage
    
    Тем не менее, невозможно определить объект типа \mintinline{cpp}{Container}. Например
    \begin{cppcode}
        Container c; // error : there can be no objects of an abstract class
        Container* p = new Vector_container(10); // OK: Container -- интерфейс
    \end{cppcode}
    
    Контейнер может служить только интерфейсом для класса, который реализует свои функции \mintinline{cpp}{operator[]()} и
    \mintinline{cpp}{size()}
    Класс с чисто виртуальной функцией называется абстрактным классом.
    
    \vspace{4mm}

    Класс контейнер может быть использован следующим образом
    \begin{cppcode}
        void use(Container& c)
        {
            const int sz = c.size();
            for (int i=0; i!=sz; ++i)
            cout << c[i] << '\n';
        }
    \end{cppcode}

    Обратите внимание, как \mintinline{cpp}{use()} использует интерфейс \mintinline{cpp}{Container} при полном незнании деталей 
    реализации. Он использует \mintinline{cpp}{size()} и \mintinline{cpp}{[]}, не имея представления о том, какой именно тип 
    обеспечивает их реализацию. Класс, который обеспечивает интерфейс для множества других классов, часто называют полиморфным типом.
    
    \vspace{4mm}
    
    Как правило для абстрактных классов, контейнер не имеет конструктора.
    Ведь у него нет данных для инициализации. С другой стороны, контейнер имеет деструктор и этот деструктор
    является виртуальным, так что классы, производные от \mintinline{cpp}{Container},
    могут предоставлять реализации. Опять же, это характерно для абстрактных классов, потому что ими, как правило, манипулируют с 
    помощью ссылок или указателей, а тот, кто уничтожает \mintinline{cpp}{Container} с помощью указателя, не знает, 
    какие ресурсы принадлежат его реализации.    
\end{frame}

\begin{frame}[fragile,allowframebreaks]{Реализация интерфейса класса \mintinline{cpp}{Container}}
    Абстрактный класс Container определяет только интерфейс и никакой реализации. 
    Чтобы контейнер был полезен, мы должны реализовать контейнер, который реализует функции, требуемые его интерфейсом. 
    Для этого мы можем использовать конкретный класс \mintinline{cpp}{Vector}:
    \begin{cppcode}
        class Vector_container : public Container { // Vector_container реализует интерфейс Container
          public:
            Vector_container(int s) : v(s) { } // Vector вектор s элементов
            ~Vector_container() {}
            double& operator[](int i) override { return v[i]; }
            int size() const override { return v.size(); }
            private:
            Vector v;
        };
    \end{cppcode}
    Выражение \mintinline{cpp}{:public} может читаться как <<является производным от>> или <<является подтипом>>. Класс 
    \mintinline{cpp}{Vector_container} называется производным от класса \mintinline{cpp}{Container}, а класс \mintinline{cpp}{Container} 
    считается базовым классом класса \mintinline{cpp}{Vector_container}. 
    Альтернативная терминология: подкласс и суперкласс \mintinline{cpp}{Vector_container} и \mintinline{cpp}{Container} соответственно.
    Говорят, что производный класс наследует членов от своего базового класса, поэтому использование базовых и производных классов 
    обычно называют наследованием.
    
     \vspace{4mm}
    
    Считается, что члены \mintinline{cpp}{operator[]()} и \mintinline{cpp}{size()} переопределяют
    соответствующие члены в базовом классе \mintinline{cpp}{Container}.
    Здесь используется явное переопределение (\mintinline{cpp}{override}), чтобы явно указать намерение. Использование 
    \mintinline{cpp}{override} является необязательным, но будучи явным, компилятор может отлавливать ошибки, такие как неправильное 
    написание имен функций или небольшие различия между типом виртуальной функции и ее предполагаемым переопределением. Явное 
    использование переопределения особенно полезно в больших классах, где иначе сложно понять, что должно быть переопределено.
    
    \vspace{4mm}
    
    Деструктор (\mintinline{cpp}{~Vector_container()}) переопределяет деструктор базового класса (\mintinline{cpp}{~Container()}). 
    Обратите внимание, что деструктор-член \mintinline{cpp}{~Vector()} неявно вызывается деструктором своего класса 
    (\mintinline{cpp}{~Vector_container()}).
    
    \vspace{4mm}
    
    Для функции \mintinline{cpp}{use(Container &)} использующей контейнер в полном незнании деталей реализации, 
    некоторые другие функции должны будут создать объект, с которым она может работать. Например:
    
    \begin{cppcode}
        void g()
        {
            Vector_container vc(10); // Vector 10 элементов
            // ... инициализация данных ...
            use(vc);
        }
    \end{cppcode}

    \newpage
    Поскольку функция \mintinline{cpp}{use()} не знает о \mintinline{cpp}{Vector_containers}, а знает только интерфейс 
    \mintinline{cpp}{Container}, она будет работать так же хорошо для другой реализации контейнера. Например:
    \begin{cppcode}
        class List_container : public Container { // List_container implements Container
          public:
            List_container() { } // пустой список
            List_container(initializer_list<double> il) : ld{il} { }
            ~List_container() {}
            double& operator[](int i) override;
            int size() const override { return ld.size(); }
          private:
            std::list<double> ld; // (standard-library) список double 
        };
        double& List_container::operator[](int i)
            {
            for (auto& x : ld) {
                if (i==0)
                    return x;
                --i;
            }
            throw out_of_range{"List container"};
        }
    \end{cppcode}
    
    Здесь представление данных -- список стандартной библиотеки параметрзованный типом \mintinline{cpp}{<double>}. Обычно не стоит
    реализовывать контейнер с операцией произвольного доступа используя список, потому что производительность такого доступа по списку 
    ужасна по сравнению с произвольным доступом у вектора. Однако здесь просто показана реализация, которая радикально отличается от 
    обычной.
    
    \vspace{4mm}
    
    Некоторая функция может создать \mintinline{cpp}{List_container} и затем использовать его при помощи функции  \mintinline{cpp}{use()}:
    \begin{cppcode}
        void h()
        {
            List_container lc = { 1, 2, 3, 4, 5, 6, 7, 8, 9 };
            use(lc);
        }
    \end{cppcode}

    
    Идея в том, что \mintinline{cpp}{use(Container &)} понятия не имеет, является ли его аргумент \mintinline{cpp}{Vector_container} , 
    \mintinline{cpp}{List_container} или каким-либо другим видом контейнера; этого не нужно знать. Функция может использовать любой вид 
    контейнера. Она знает только интерфейс, определенный контейнером. Следовательно, \mintinline{cpp}{use(Container &)} не нужно 
    перекомпилировать, если используется реализация \mintinline{cpp}{List_container} или используется совершенно новый класс, 
    производный от \mintinline{cpp}{Container}.
    
    \vspace{4mm}
    
    Обратная сторона этой гибкости заключается в том, что объектами нужно манипулировать с помощью указателей или ссылок.
    
\end{frame}

\section{Виртуальные функции}

\begin{frame}[fragile,allowframebreaks]{Виртуальные функции}
    Рассмотрим снова использование контейнера:
    \begin{cppcode}
        void use(Container& c)
        {
            const int sz = c.size();
            for (int i=0; i!=sz; ++i)
                cout << c[i] << '\n';
        }
    \end{cppcode}
    Как разрешается вызов \mintinline{cpp}{c[i]} в функции \mintinline{cpp}{use(Container &)} до правильного оператора 
    \mintinline{cpp}{operator[]()}? Когда \mintinline{cpp}{h()}  вызывает \mintinline{cpp}{use()}, должен быть вызван оператор 
    \mintinline{cpp}{operator[]()}, который реализован в \mintinline{cpp}{List_container}. Когда \mintinline{cpp}{g()}  вызывает 
    \mintinline{cpp}{use()}, должен быть вызван оператор \mintinline{cpp}{operator[]()}, который реализован в 
    \mintinline{cpp}{Vector_container}.   
    
    \vspace{4mm}
     
    Для достижения этого разрешения объект-контейнер должен содержать информацию, позволяющую ему выбрать правильную функцию для вызова 
    во время выполнения. Обычный метод реализации заключается в том, что компилятор преобразует имя виртуальной функции в индекс таблицы 
    указателей на функции. Эта таблица обычно называется таблицей виртуальных функций или просто \mintinline{cpp}{vtbl}.
    
    \vspace{4mm}
    
    Каждый класс с виртуальными функциями имеет свою собственную \mintinline{cpp}{vtbl}, регистрирующую его виртуальные функции. Это может быть представлено графически следующим образом:
    
    \begin{figure}
        \small
        \begin{tikzpicture}[align=center,font=\ttfamily]
        
        \node (vectorContainer) at (0,0) {\mintinline{cpp}{Vector_container:}};
        \node[draw,rectangle,dashed,below = 0mm of vectorContainer,minimum height=5mm, minimum width = 15mm] (containerV) {};
        \node[draw,rectangle,below = 0mm of vectorContainer,minimum height=10mm, minimum width = 15mm] (containerVFrame) {};
        \node[below=0 of containerV, minimum height=5mm] (v) {\mintinline{cpp}{v}};
        
        \node[draw,rectangle,right = 10mm of containerV,minimum height=5mm, minimum width = 15mm] (vtblV1) {};
        \node[draw,rectangle,below = 0 of vtblV1,minimum height=5mm, minimum width = 15mm] (vtblV2) {};
        \node[draw,rectangle,below = 0 of vtblV2,minimum height=5mm, minimum width = 15mm] (vtblV3) {};
        \node[above=0 of vtblV1] (vtbl1) {\mintinline{cpp}{vtbl:}};
        
        \node[draw,rectangle,right = 10mm of vtblV1,minimum height=5mm, minimum width = 57mm, align=center] (explicitV1) {\mintinline{cpp}{Vector_container::operator[]()}};        
        \node[draw,rectangle,below = 4mm of explicitV1,minimum height=5mm, minimum width = 57mm, align=center] (explicitV2) {\mintinline{cpp}{Vector_container::size()}};
        \node[draw,rectangle,below = 4mm of explicitV2,minimum height=5mm, minimum width = 57mm, align=center] (explicitV3) {\mintinline{cpp}{Vector_container::~Vector_container()}};
        
        \node[below = 15mm of containerVFrame] (listContainer) {\mintinline{cpp}{List_container:}};
        \node[draw,rectangle,dashed,below = 0 of listContainer,minimum height=5mm, minimum width = 15mm] (containerL) {};
        \node[draw,rectangle,below = 0 of listContainer,minimum height=15mm, minimum width = 15mm] (containerLFrame) {};
        \node[below=0 of containerL, minimum height=10mm] (ld) {\mintinline{cpp}{ld}};
        
        \node[draw,rectangle,right = 10mm of containerL,minimum height=5mm, minimum width = 15mm] (vtblL1) {};
        \node[draw,rectangle,below = 0 of vtblL1,minimum height=5mm, minimum width = 15mm] (vtblL2) {};
        \node[draw,rectangle,below = 0 of vtblL2,minimum height=5mm, minimum width = 15mm] (vtblL3) {};
        \node[above=0 of vtblL1] (vtbl2) {\mintinline{cpp}{vtbl:}};
        
        \node[draw,rectangle,right = 10mm of vtblL1,minimum height=5mm, minimum width = 57mm, align=center] (explicitL1) {\mintinline{cpp}{List_container::operator[]()}};        
        \node[draw,rectangle,below = 4mm of explicitL1,minimum height=5mm, minimum width = 57mm, align=center] (explicitL2) {\mintinline{cpp}{List_container::size()}};
        \node[draw,rectangle,below = 4mm of explicitL2,minimum height=5mm, minimum width = 57mm, align=center] (explicitL3) {\mintinline{cpp}{List_container::~Vector_container()}};
        
        \path[draw,-stealth] (containerV.center) -- (vtblV1.west);
        \path[draw,-stealth] (vtblV1.center) -- (explicitV1.west);
        \path[draw,-stealth] (vtblV2.center) -- (explicitV2.west);
        \path[draw,-stealth] (vtblV3.center) -- (explicitV3.west);
        
        \path[draw,-stealth] (containerL.center) -- (vtblL1.west);
        \path[draw,-stealth] (vtblL1.center) -- (explicitL1.west);
        \path[draw,-stealth] (vtblL2.center) -- (explicitL2.west);
        \path[draw,-stealth] (vtblL3.center) -- (explicitL3.west);
                
        \end{tikzpicture}
    \end{figure}
    Функции в \mintinline{cpp}{vtbl} позволяют правильно использовать объект, даже если размер объекта и расположение его данных неизвестны вызывающей стороне. Реализация вызывающей стороны должна знать только местоположение указателя на \mintinline{cpp}{vtbl} в \mintinline{cpp}{Container} и индекс, используемый для каждой виртуальной функции. Этот механизм виртуального вызова можно сделать почти таким же эффективным, как механизм «обычного вызова функции» (в пределах 25\%). Его служебная память занимает один указатель в каждом объекте класса с виртуальными функциями плюс одна \mintinline{cpp}{vtbl} для каждого такого класса.
\end{frame}







\end{document}