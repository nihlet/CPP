% !TeX encoding = UTF-8
% !TeX spellcheck = ru_RU
% !TeX root = lecture3.tex

\documentclass[
    9pt,
    hyperref={pdfencoding=unicode}
    ]{beamer}

\usepackage{polyglossia}   %% загружает пакет многоязыковой вёрстки

\setmainlanguage{russian}  %% устанавливает главный язык документа
\setmainfont{Times New Roman}
\setromanfont{Times New Roman} 
\setmonofont{Fira Code} 
\setsansfont{Arial} 

\PolyglossiaSetup{russian}{indentfirst=true}

\newfontfamily{\cyrillicfont}{Times New Roman}[Ligatures=TeX]
\newfontfamily{\cyrillicfontrm}{Times New Roman}[Ligatures=TeX]
\newfontfamily{\cyrillicfonttt}{Fira Code}[Ligatures=TeX]
\newfontfamily{\cyrillicfontsf}{Arial}[Ligatures=TeX]

\usepackage{wasysym,amssymb,amsthm}

\usepackage{pgfplots}
\usetikzlibrary{arrows,positioning,shapes}

\usepackage{minted}
\newminted{cpp}{frame=leftline,framerule=1.5pt,framesep=1em,breaklines=true,autogobble=true}

\usetheme{default}
\usecolortheme{seagull}
\useoutertheme[subsection=false]{miniframes}
\setbeamertemplate{mini frames}[tick]

\usepackage{ragged2e}
\usepackage{xcolor}

\theoremstyle{definition}
\newtheorem{myDef}{Определение}

\usepackage{csquotes}
\usepackage[%
backend=biber,% движок
bibencoding=utf8,% кодировка bib файла
sorting=none,% настройка сортировки списка литературы
style=gost-numeric,% стиль цитирования и библиографии (по ГОСТ)
language=autobib,% получение языка из babel/polyglossia, default: autobib % если ставить autocite или auto, то цитаты в тексте с указанием страницы, получат указание страницы на языке оригинала
autolang=other,% многоязычная библиография
clearlang=true,% внутренний сброс поля language, если он совпадает с языком из babel/polyglossia
defernumbers=true,% нумерация проставляется после двух компиляций, зато позволяет выцеплять библиографию по ключевым словам и нумеровать не из большего списка
sortcites=true,% сортировать номера затекстовых ссылок при цитировании (если в квадратных скобках несколько ссылок, то отображаться будут отсортированно, а не абы как)
doi=false,% Показывать или нет ссылки на DOI
isbn=true,% Показывать или нет ISBN, ISSN, ISRN
]{biblatex}
\addbibresource{biblio.bib}

\makeatletter
\setbeamertemplate{section in head/foot}{%
    \parbox[c][0.33cm][t]{\dimexpr(\textwidth-1.3cm)/\beamer@sectionmax\relax}{%\
        \hyphenpenalty=9999\RaggedRight\fontsize{4}{4}\selectfont\insertsectionhead}}
\makeatother

\author{Воронин Андрей Андреевич}
\title{Лекция 3 – Указатели, массивы, ссылки и вектора}
\institute{Кафедра прикладной математики и информатики}
\date{\today}
\begin{document}
\titlepage 

\section{Синтаксис}

\begin{frame}[fragile]{Введение}
    \begin{myDef}
        \textbf{Массив} -- непрерывно расположенная в памяти последовательность элементов одного типа.
    \end{myDef}
    
    Массив из 6 элементов типа \mintinline{cpp}{char}:
    \begin{cppcode}
        char v[6];
    \end{cppcode}

    Указатель на элемент типа \mintinline{cpp}{char}:
    \begin{cppcode}
        char* p;
    \end{cppcode}
    
    Взятие ссылки и разыменовывание указателя:
    \begin{cppcode}
        char *p = &v[3]; // p указывает на 4 элемент массива 
        char x = *p; // разыменование указателя -- получаем значение на которое указывает указатель
    \end{cppcode}
    
    Префиксный оператор \mintinline{cpp}{&} является оператором взятия ссылки.
\end{frame}

\begin{frame}{Указатель и массив}
    \begin{figure}
        \begin{tikzpicture}[align=center,font=\ttfamily]
            \node (p) at (0,0) {\textbf{p:}};
            \node[draw,rectangle,right = of p,minimum height=5mm, minimum width = 10mm] (pp) {};
            
            \node[below = 20mm of p] (v) {\textbf{v:}};
            \node[draw,rectangle,right = of v,minimum height=5mm, minimum width = 10mm] (v0) {};
            \node[draw,rectangle,right = 0 of v0,minimum height=5mm, minimum width = 10mm] (v1) {};
            \node[draw,rectangle,right = 0 of v1,minimum height=5mm, minimum width = 10mm] (v2) {};
            \node[draw,rectangle,right = 0 of v2,minimum height=5mm, minimum width = 10mm] (v3) {};
            \node[draw,rectangle,right = 0 of v3,minimum height=5mm, minimum width = 10mm] (v4) {};
            \node[draw,rectangle,right = 0 of v4,minimum height=5mm, minimum width = 10mm] (v5) {};
            
            \node[above =1mm of v0] (n0) {0:};
            \node[above =1mm of v1] (n1) {1:};
            \node[above =1mm of v2] (n2) {2:};
            \node[above =1mm of v3] (n3) {3:};
            \node[above =1mm of v4] (n4) {4:};
            \node[above =1mm of v5] (n5) {5:};
            
            \path[draw,-stealth] (pp.south) -- (v3.north);
        \end{tikzpicture}
    \end{figure}
\end{frame}

\begin{frame}[fragile]{Зачем нужны массивы?}
    \begin{cppcode}
        // Выделяем 30 целочисленных переменных (каждая с разным именем)
        int testResultStudent1;
        int testResultStudent2;
        int testResultStudent3;
        // ...
        int testResultStudent30;
    \end{cppcode}
    
    \vspace{2em}
    \begin{cppcode}
        int testResult[30]; // выделяем 30 целочисленных переменных, используя фиксированный массив
    \end{cppcode}
\end{frame}


\begin{frame}[fragile]{Доступ к элементам массива}
    \begin{cppcode}
        #include <iostream>
        
        int main()
        {
            int array[5]; // массив из пяти чисел
            array[0] = 3; // индекс первого элемента - 0 (нулевой элемент)
            array[1] = 2;
            array[2] = 4;
            array[3] = 8;
            array[4] = 12; // индекс последнего элемента - 4 
            
            std::cout << "The lowest number is " << array[0] << "\n";
            std::cout << "The sum of the first 5 numbers is " << array[0] + array[1] + array[2] + array[3] + array[4] << "\n";
            
            return 0;
        }
    \end{cppcode}
\end{frame}

\begin{frame}[fragile]{Пример 1}
    \begin{cppcode}
        void copy_for_some_purpose()
        {
            int v1[10] = {0,1,2,3,4,5,6,7,8,9};
            int v2[10];
            
            for(auto i=0; i!=10; ++i)
                v2[i] = v1[i];
            // ...
        }
    \end{cppcode}
\end{frame}


\begin{frame}[fragile]{Пример 2}
    \begin{cppcode}
        void print()
        {
            int v[] = {0,1,2,3,4,5,6,7,8,9};
            
            for (auto x : v)
                cout << x << '\n';
                
            for (auto x : {12,73,08,23,22})
                cout<< x << '\n';    
        }
    \end{cppcode}
\end{frame}

\begin{frame}[fragile]{Пример 3}
    \begin{cppcode}
        void increment()
        {
            int v[] = {0,1,2,3,4,5,6,7,8,9};
            
            for (auto &x : v)
                ++x;
        }
    \end{cppcode}
\end{frame}

\begin{frame}[fragile]{Ссылки}
    Во время объявления переменной префикс \mintinline{cpp}{&} означает "сыллка на".
    
    Ссылка похожа на указатель, за тем исключением, что нет необходимости разыменовывать указатель для доступа к значению. Также нельзя сменить объект на который она указывает.
    
    Ссылки в первую очередь важны для определения аргументов функции:
    \begin{cppcode}
        void sort(vector<double>& v); // сортируем вектор v (v это вектор значений double)
    \end{cppcode}

    \vspace{2em}

    Объявляя \mintinline{cpp}{v} как ссылку мы изменяем поведение функции, запрещая копирование аргументов.
    
    \vspace{2em}
    
    Константная ссылка кроме того, гарантирует неизменяемость аргументов внутри функции.
    \begin{cppcode}
        double sum(const vector<double> &v);
    \end{cppcode}

    
\end{frame}


\begin{frame}[fragile]{Нулевой указатель}
    \begin{cppcode}
        double *pd = nullptr;
        Link<Record> *lst = nullptr; // указатель на параметризуемый контейнер
        int x = nullptr;     		 // error: nullptr is a pointer not an integer
    \end{cppcode}

    \vspace{2em}
    
    В более раннем коде можно встретить использование \mintinline{cpp}{0} или \mintinline{cpp}{NULL} вместо использования \mintinline{cpp}{nullptr}. Тем не менее использование \mintinline{cpp}{nullptr} избавляет от коллизии с целочисленными литералами и указателями.  

\end{frame}

\begin{frame}[fragile]{Нулевой указатель}
    Часто важно проверять аргументы функции на тот случай, что они на самом деле указывают на существующий объект.
    \begin{cppcode}
        int count_x (const char *p, char x)
        {
            if (p==nullptr)
                return 0;
            int count = 0;
            for (; *p!=0; ++p)
                if (*p==x)
                    ++count;
            return count;
        }
    \end{cppcode}
    Выше мы предполагаем, что \mintinline{cpp}{char *} это так называемая строка в C-стиле, которая оканчивается нулевым терминирующим символом \mintinline{cpp}{'\0'}.
\end{frame}

\begin{frame}[fragile]{Нулевой указатель}
    В предыдущем примере мы не использовали инициализирующее значение в цикле \mintinline{cpp}{for}, мы можем использовать цикл \mintinline{cpp}{while}
    \begin{cppcode}
        int count_x (const char *p, char x)
        {
            if (p==nullptr)
                return 0;
            int count = 0;
            while (*p)
            {
                if (*p==x)
                    ++count;
                ++p;
            }
            return count;
        }
    \end{cppcode}
\end{frame}

\begin{frame}{Использование аппаратных возможностей}
    \begin{figure}
        \begin{tikzpicture}[align=center,font=\ttfamily]
        \node (p) at (0,0) {\textbf{p:}};
        \node[draw,rectangle,right = of p,minimum height=5mm, minimum width = 10mm] (pp) {};
        
        \node[below = 20mm of p] (v) {\textbf{v:}};
        \node[draw,rectangle,right = of v,minimum height=5mm, minimum width = 10mm] (v0) {};
        \node[draw,rectangle,right = 0 of v0,minimum height=5mm, minimum width = 10mm] (v1) {};
        \node[draw,rectangle,right = 0 of v1,minimum height=5mm, minimum width = 10mm] (v2) {};
        \node[draw,rectangle,right = 0 of v2,minimum height=5mm, minimum width = 10mm] (v3) {};
        \node[draw,rectangle,right = 0 of v3,minimum height=5mm, minimum width = 10mm] (v4) {};
        \node[draw,rectangle,right = 0 of v4,minimum height=5mm, minimum width = 10mm] (v5) {};
        
        \node[above =1mm of v0] (n0) {0:};
        \node[above =1mm of v1] (n1) {1:};
        \node[above =1mm of v2] (n2) {2:};
        \node[above =1mm of v3] (n3) {3:};
        \node[above =1mm of v4] (n4) {4:};
        \node[above =1mm of v5] (n5) {5:};
        
        \path[draw,-stealth] (pp.south) -- (v3.north);
        \end{tikzpicture}
    \end{figure}
\end{frame}

\section{Присваивание}
\begin{frame}[fragile]{Присваивание переменных}
    \begin{cppcode}
        int x = 2;
        int y = 3;
        x = y; // x становится равным 3
        // Note: x==y
    \end{cppcode}
    \begin{figure}
        \begin{tikzpicture}[align=center,node distance=1mm,font=\ttfamily]
        \node (x) at (0,0) {\textbf{x:}};
        \node[draw,rectangle,right = of x,minimum height=5mm, minimum width = 10mm] (xx) {2};
        
        \node[right = 8mm of xx] (y) {\textbf{y:}};
        \node[draw,rectangle,right = of y,minimum height=5mm, minimum width = 10mm] (yy) {3};
        
        \node[right = 8mm of yy] (assign) {\textbf{x=y;}};
        
        \node[right = 8mm of assign] (x2) {\textbf{x:}};
        \node[draw,rectangle,right = of x2,minimum height=5mm, minimum width = 10mm] (xx2) {3};
        
        \node[right = 8mm of xx2] (y2) {\textbf{y:}};
        \node[draw,rectangle,right = of y2,minimum height=5mm, minimum width = 10mm] (yy2) {3};
        \end{tikzpicture}
    \end{figure}
    
    \vspace{2em}
    если после этого \mintinline{cpp}{x=99;}, то из-за этого значение \mintinline{cpp}{y} останется равным 3.
\end{frame}


\begin{frame}[fragile]{Указатели}
    \begin{cppcode}
        int x = 2;
        int y = 3;
        int *p = &x;
        int *q = &y; // сейчас p != q и *p != *q
        p = q;       // p становится равным &y; сейчас p == q и *p == *q
    \end{cppcode}

    \vspace{2em}
    \begin{figure}
        \begin{tikzpicture}[align=center,node distance=1mm,font=\ttfamily]
        \node (p) at (0,0) {\textbf{p:}};
        \node[draw,rectangle,right = of p,minimum height=5mm, minimum width = 10mm] (pp) {88};
        
        \node[below = 10mm of p] (x) {\textbf{x:}};
        \node[draw,rectangle,right = of x,minimum height=5mm, minimum width = 10mm] (xx) {2};
        
        \node[right = 8mm of pp] (q) {\textbf{q:}};
        \node[draw,rectangle,right = of q,minimum height=5mm, minimum width = 10mm] (qq) {92};
        
        \node[right = 8mm of xx] (y) {\textbf{y:}};
        \node[draw,rectangle,right = of y,minimum height=5mm, minimum width = 10mm] (yy) {3};
        
        \node[above right = 5mm and 8mm of yy.east] (assign) {\textbf{p=q;}};
        
        \node[above right = 5mm and 8mm of assign.east] (p2) {\textbf{p:}};
        \node[draw,rectangle,right = of p2,minimum height=5mm, minimum width = 10mm] (pp2) {92};
        
        \node[below = 10mm of p2] (x2) {\textbf{x:}};
        \node[draw,rectangle,right = of x2,minimum height=5mm, minimum width = 10mm] (xx2) {2};
        
        \node[right = 8mm of pp2] (q2) {\textbf{q:}};
        \node[draw,rectangle,right = of q2,minimum height=5mm, minimum width = 10mm] (qq2) {92};
        
        \node[right = 8mm of xx2] (y2) {\textbf{y:}};
        \node[draw,rectangle,right = of y2,minimum height=5mm, minimum width = 10mm] (yy2) {3};
        
        \path[draw,-stealth] (pp) -- (xx);
        \path[draw,-stealth] (qq) -- (yy);
        \path[draw,-stealth] (pp2) -- (yy2);
        \path[draw,-stealth] (qq2) -- (yy2);
        \end{tikzpicture}
    \end{figure}
\end{frame}

\begin{frame}[fragile]{Ссылки}
    \begin{cppcode}
        int x = 2;
        int y = 3;
        int &r = x; // r ссылается на x
        int &t = y; // t ссылается на y
        r = t;      // читаем по ссылке t и записываем по ссылке r
    \end{cppcode}
    
    \vspace{2em}
    \begin{figure}
        \begin{tikzpicture}[align=center,node distance=1mm,font=\ttfamily]
        \node (p) at (0,0) {\textbf{r:}};
        \node[draw,rectangle,right = of p,minimum height=5mm, minimum width = 10mm] (pp) {88};
        
        \node[below = 10mm of p] (x) {\textbf{x:}};
        \node[draw,rectangle,right = of x,minimum height=5mm, minimum width = 10mm] (xx) {2};
        
        \node[right = 8mm of pp] (q) {\textbf{t:}};
        \node[draw,rectangle,right = of q,minimum height=5mm, minimum width = 10mm] (qq) {92};
        
        \node[right = 8mm of xx] (y) {\textbf{y:}};
        \node[draw,rectangle,right = of y,minimum height=5mm, minimum width = 10mm] (yy) {3};
        
        \node[above right = 5mm and 8mm of yy.east] (assign) {\textbf{r=t;}};
        
        \node[above right = 5mm and 8mm of assign.east] (p2) {\textbf{r:}};
        \node[draw,rectangle,right = of p2,minimum height=5mm, minimum width = 10mm] (pp2) {88};
        
        \node[below = 10mm of p2] (x2) {\textbf{x:}};
        \node[draw,rectangle,right = of x2,minimum height=5mm, minimum width = 10mm] (xx2) {3};
        
        \node[right = 8mm of pp2] (q2) {\textbf{t:}};
        \node[draw,rectangle,right = of q2,minimum height=5mm, minimum width = 10mm] (qq2) {92};
        
        \node[right = 8mm of xx2] (y2) {\textbf{y:}};
        \node[draw,rectangle,right = of y2,minimum height=5mm, minimum width = 10mm] (yy2) {3};
        
        \path[draw,-stealth] (pp) -- (xx);
        \path[draw,-stealth] (qq) -- (yy);
        \path[draw,-stealth] (pp2) -- (xx2);
        \path[draw,-stealth] (qq2) -- (yy2);
        \end{tikzpicture}
    \end{figure}
\end{frame}

\begin{frame}[fragile]{Инициализация}
    \begin{cppcode}
        int x = 7;
        int &r {x}; // связываем r с x (r ссылается на x)
        r = 8;      // присваеваем значение к чему-то на что ссылается r
        
        int &r2;    // error: неинициализированная ссылка
        r2 = 99;    // присваеваем значение к чему-то на что ссылается r2
    \end{cppcode}
\end{frame}

\section{Вектор}
\begin{frame}[fragile]{Инициализация}
    \begin{cppcode}
        #include <vector>
        std::vector<int> myVector; // создаем пустой вектор типа int
        myVector.reserve(10);      // резервируем память под 10 элементов типа int
        
        int myArray[10]; // аналог в виде массива 
    \end{cppcode}

    \vspace{1em}
    \begin{cppcode}
        #include <iostream>
        #include <vector> 
        using namespace std;
        int main()
        {
            vector<int> myVector(10);   // объявляем вектор размером в 10 элементов и инициализируем их нулями
            // вывод элементов вектора на экран
            for(int i = 0; i < myVector.size(); i++)
                cout << myVector[i] << ' ';
            return 0;
        }
    \end{cppcode}
\end{frame}


\begin{frame}[fragile]{Инициализация}
    \begin{cppcode}
        #include <iostream>
        #include <vector>
        using namespace std;
        int main()
        {
            vector<int> vec1(3);
            // инициализируем элементы вектора vec1
            vec1[0] = 4;
            vec1[1] = 2;
            vec1[2] = 1;
            vector<int> vec2(3);
            // инициализируем элементы вектора vec2
            vec2[2] = 1;
            vec2[1] = 2;
            vec2[0] = 4;
            // сравниваем массивы
            if (vec1 == vec2) {
                cout << "vec1 == vec2" << endl;
            }
            return 0;
        }
    \end{cppcode}
\end{frame}

\begin{frame}[fragile]{Вектор}
    \begin{cppcode}
        #include <iostream>
        #include <vector>
        #include <iterator> // заголовочный файл итераторов
        using namespace std;
        
        int main()
        {
            vector<int> vec1; // создаем пустой вектор
            // добавляем в конец вектора vec1 элементы 4, 3, 1
            vec1.insert(vec1.end(), 4);
            vec1.insert(vec1.end(), 3);
            vec1.insert(vec1.end(), 1);
            // вывод на экран элементов вектора
            copy( vec1.begin(),   // итератор начала массива
                  vec1.end(),     // итератор конца массива
                  ostream_iterator<int>(cout," ")   //итератор потока вывода
            );
            return 0;
        }
    \end{cppcode}
\end{frame}

\section{Рекомендации}
\begin{frame}[t,allowframebreaks]{Советы}
    \begin{enumerate}
        \item Не паникуй!\textcopyright  \: Все станет яснее со временем.
        \item Не используй только встроенные фичи в язык -- кроме этого еще есть стандартная библиотека.
        \item Нет необходимости знать все тонкости языка для того чтобы писать хорошие программы.
        \item "Пакуй" целостные группы операций в функции.
        \item Функция должна отвечать за одну логическую операцию.
        \item Старайся создавать функции короткими.
        \item Используй перегрузку когда функции производят одну и ту же операцию над различными типами.
        \item Если функция может быть рассчитана на этапе компиляции, то используй \mintinline{cpp}{constexpr}. 
        \item Необходимо понимать как языковые примитивы отображаются на аппаратное обеспечение.
        \item 300000000 хуже чем 300'000'000.
        \item Избегай сложных выражений.
        \item Избегай преобразований с округлением.
        \item Минимизируй область видимости переменной.
        \item Избегай "магических" констант.
        \item Предпочитай неизменяемые данные.
        \item Объявляй только одно имя переменной в строчке.
        \item Локальные переменные и распространенные переменные должны иметь короткие имена, а необычные переменные объявленные далеко от вычислений должны иметь длинные имена.
        \item Избегай имен похожих друг на друга.
        \item Избегай имен вида \mintinline{cpp}{ALL_CAPS}.
        \item Предпочитай инициализацию в виде \mintinline{cpp}{int x {7.9};}
        \item Используй \mintinline{cpp}{auto} чтобы избегать повторения имен типов.
        \item Избегай неинициализированных переменных.
        \item Не объявляй переменную до того как можешь ее инициализировать.
        \item \mintinline{cpp}{unsigned} типы используются только для битовых операций.
        \item Используй указатели просто и прямолинейно.
        \item Не пиши комментариев смысл которых понятен из кода.
        \item Используй \mintinline{cpp}{nullptr} вместо \mintinline{cpp}{0} или \mintinline{cpp}{NULL}.
        \item Объясняй свои намерения в комментариях.
        \item Поддерживай однородную табуляцию.
    \end{enumerate}
\end{frame}

\begin{frame}[t,allowframebreaks]
    \frametitle{Литературные источники}
    
    \nocite{*}
    \printbibliography
\end{frame}


\end{document}