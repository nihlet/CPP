% !TeX encoding = UTF-8
% !TeX spellcheck = ru_RU
% !TeX root = lecture4.tex

\documentclass[
    9pt,
    hyperref={pdfencoding=unicode}
    ]{beamer}

\usepackage{polyglossia}   %% загружает пакет многоязыковой вёрстки

\setmainlanguage{russian}  %% устанавливает главный язык документа
\setmainfont{Liberation Serif}
\setromanfont{Liberation Serif} 
\setmonofont{Fira Code} 
\setsansfont{Liberation Sans} 

\PolyglossiaSetup{russian}{indentfirst=true}

\newfontfamily{\cyrillicfont}{Liberation Serif}[Ligatures=TeX]
\newfontfamily{\cyrillicfontrm}{Liberation Serif}[Ligatures=TeX]
\newfontfamily{\cyrillicfonttt}{Fira Code}[Ligatures=TeX]
\newfontfamily{\cyrillicfontsf}{Liberation Sans}[Ligatures=TeX]

\usepackage{wasysym,amssymb,amsthm}

\usepackage{pgfplots}
\usetikzlibrary{arrows,positioning,shapes}

\usepackage{minted}
\newminted{cpp}{frame=leftline,framerule=1.5pt,framesep=1em,breaklines=true,autogobble=true,tabsize=4}

\usetheme{default}
\usecolortheme{seagull}
\useoutertheme[subsection=false]{miniframes}
\setbeamertemplate{mini frames}[tick]

\usepackage{ragged2e}
\usepackage{xcolor}

\theoremstyle{definition}
\newtheorem{myDef}{Определение}

\usepackage{csquotes}
\usepackage[%
backend=biber,% движок
bibencoding=utf8,% кодировка bib файла
sorting=none,% настройка сортировки списка литературы
style=gost-numeric,% стиль цитирования и библиографии (по ГОСТ)
language=autobib,% получение языка из babel/polyglossia, default: autobib % если ставить autocite или auto, то цитаты в тексте с указанием страницы, получат указание страницы на языке оригинала
autolang=other,% многоязычная библиография
clearlang=true,% внутренний сброс поля language, если он совпадает с языком из babel/polyglossia
defernumbers=true,% нумерация проставляется после двух компиляций, зато позволяет выцеплять библиографию по ключевым словам и нумеровать не из большего списка
sortcites=true,% сортировать номера затекстовых ссылок при цитировании (если в квадратных скобках несколько ссылок, то отображаться будут отсортированно, а не абы как)
doi=false,% Показывать или нет ссылки на DOI
isbn=true,% Показывать или нет ISBN, ISSN, ISRN
]{biblatex}
\addbibresource{biblio.bib}

\makeatletter
\setbeamertemplate{section in head/foot}{%
    \parbox[c][0.33cm][t]{\dimexpr(\textwidth-1.3cm)/\beamer@sectionmax\relax}{%\
        \hyphenpenalty=9999\RaggedRight\fontsize{4}{4}\selectfont\insertsectionhead}}
\makeatother

\author{Воронин Андрей Андреевич}
\title{Лекция 4 – Статические многомерные массивы, динамическое выделение памяти, динамические многомерные массивы}
\institute{Кафедра прикладной математики и информатики}
\date{\today}
\begin{document}
\titlepage 
\section{Статические многомерные массивы}

\begin{frame}[fragile]{Инициализация}
    Инициализация одномерного массива
    \begin{cppcode}
        const int DIM1 = 4;
        int array[DIM1] = {0,1,2,3}
    \end{cppcode}
    
    Инициализация двумерного массива
    \begin{cppcode}
        const int DIM1 = 3;
        const int DIM2 = 5;
        
        int array[DIM1][DIM2] = {
            { 1, 2, 3, 4, 5 },
            { 2, 4, 6, 8, 10 },
            { 3, 6, 9, 12, 15 }
        };
    \end{cppcode}
    
\end{frame}

\begin{frame}[fragile]{Ввод и вывод значений массива}
    \footnotesize
    \begin{columns}
        \begin{column}{0.7\textwidth}
             \begin{cppcode}
                #include <iostream>        
                using namespace std;   
                
                const int DIM1 = 3;
                const int DIM2 = 5;
                
                int array[DIM1][DIM2];
                
                int main() {            
                    for (int i = 0; i < DIM1; i++) {
                        for (int j = 0; j < DIM2; j++) {
                            array[i][j] = (i + 1) * 10 + (j + 1);
                        }
                    }
                    
                    for (int i = 0; i < DIM1; i++) {
                        for (int j = 0; j < DIM2; j++) {
                            cout << array[i][j] << "\t";
                        }
                        cout << endl;
                    }
                    
                    return 0;
                }
            \end{cppcode}
        \end{column}
        \begin{column}{0.3\textwidth}
            \begin{minted}{bash}
11  12  13  14  15
21  22  23  24  25
31  32  33  34  35
            \end{minted}
        \end{column}
    \end{columns}
   
\end{frame}


\begin{frame}[fragile]{Расположение в памяти}
    \begin{cppcode}
        #include <iostream>
        
        int main()
        {
            char v[] = {'a', 'b', 'c', 'd', 'e', 'f'};
            char *p = &v[0];
            
            if (v == p)
                std::cout << "p == v";
            return 0;
        }
    \end{cppcode}
    \begin{figure}
        \begin{tikzpicture}[align=center,font=\ttfamily]
        \node (p) at (0,0) {\textbf{p:}};
        \node[draw,rectangle,right = of p,minimum height=5mm, minimum width = 10mm] (pp) {};
        
        \node[below = 20mm of p] (v) {\textbf{v:}};
        \node[draw,rectangle,right = of v,minimum height=5mm, minimum width = 10mm] (v0) {a};
        \node[draw,rectangle,right = 0 of v0,minimum height=5mm, minimum width = 10mm] (v1) {b};
        \node[draw,rectangle,right = 0 of v1,minimum height=5mm, minimum width = 10mm] (v2) {c};
        \node[draw,rectangle,right = 0 of v2,minimum height=5mm, minimum width = 10mm] (v3) {d};
        \node[draw,rectangle,right = 0 of v3,minimum height=5mm, minimum width = 10mm] (v4) {e};
        \node[draw,rectangle,right = 0 of v4,minimum height=5mm, minimum width = 10mm] (v5) {f};
        
        \node[above =1mm of v0] (n0) {0:};
        \node[above =1mm of v1] (n1) {1:};
        \node[above =1mm of v2] (n2) {2:};
        \node[above =1mm of v3] (n3) {3:};
        \node[above =1mm of v4] (n4) {4:};
        \node[above =1mm of v5] (n5) {5:};
        
        \path[draw,-stealth] (pp.south) -- (v0.north);
        \end{tikzpicture}
    \end{figure}
\end{frame}


\begin{frame}[fragile]{Расположение в памяти}
    \begin{cppcode}
            int a[2][3] = {{1,2,3},{4,5,6}};
            
            std::cout << a << std::endl;        //0x7ffdd91484e0
            std::cout << a[0] << std::endl;     //0x7ffdd91484e0
            std::cout << a[1] << std::endl;     //0x7ffdd91484ec
            std::cout << &a[0][0] << std::endl; //0x7ffdd91484e0
            std::cout << &a[1][0] << std::endl; //0x7ffdd91484ec
    \end{cppcode}
    \begin{figure}
        \begin{tikzpicture}[align=center,font=\ttfamily]
        \node[draw,minimum height=5mm, minimum width = 10mm] (a0) {1};
        \node[draw,rectangle,right = 0 of a0,minimum height=5mm, minimum width = 10mm] (a1) {2};
        \node[draw,rectangle,right = 0 of a1,minimum height=5mm, minimum width = 10mm] (a2) {3};
        \node[draw,rectangle,right = 0 of a2,minimum height=5mm, minimum width = 10mm] (a3) {4};
        \node[draw,rectangle,right = 0 of a3,minimum height=5mm, minimum width = 10mm] (a4) {5};
        \node[draw,rectangle,right = 0 of a4,minimum height=5mm, minimum width = 10mm] (a5) {6};
        
   
        \node[draw,rectangle,above = of a0,minimum height=5mm, minimum width = 10mm] (b0) {};  
        \node[draw,rectangle,above = of a3,minimum height=5mm, minimum width = 10mm] (b1) {};
        
        \node[left = 1mm of a0] (al) {\textbf{a:}};
        \node[left = 1mm of b0] (bl1) {\textbf{a[0]:}};
        \node[left = 1mm of b1] (bl2) {\textbf{a[1]:}};
        
             
        \path[draw,-stealth] (b0.south) -- (a0.north);
        \path[draw,-stealth] (b1.south) -- (a3.north);
        \end{tikzpicture}
    \end{figure}
\end{frame}


\begin{frame}[fragile]{Расположение в памяти}
    \begin{cppcode}
        int a[2][3] = {{1,2,3},{4,5,6}};
        
        int v1 = a[1][1];
        int v2 = *(*(a+1)+1);
        
        int p = &a[1][1];
        if (v1 == v2) std::cout << "v1 == v2";
    \end{cppcode}
    \begin{figure}
        \begin{tikzpicture}[align=center,font=\ttfamily]
        \node[draw,minimum height=5mm, minimum width = 10mm] (a0) {1};
        \node[draw,rectangle,right = 0 of a0,minimum height=5mm, minimum width = 10mm] (a1) {2};
        \node[draw,rectangle,right = 0 of a1,minimum height=5mm, minimum width = 10mm] (a2) {3};
        \node[draw,rectangle,right = 0 of a2,minimum height=5mm, minimum width = 10mm] (a3) {4};
        \node[draw,rectangle,right = 0 of a3,minimum height=5mm, minimum width = 10mm] (a4) {5};
        \node[draw,rectangle,right = 0 of a4,minimum height=5mm, minimum width = 10mm] (a5) {6};
        
        
        \node[draw,rectangle,above = of a0,minimum height=5mm, minimum width = 10mm] (b0) {};  
        \node[draw,rectangle,above = of a3,minimum height=5mm, minimum width = 10mm] (b1) {};
        \node[draw,rectangle,below = of a4,minimum height=5mm, minimum width = 10mm] (p) {};
        
        \node[left = 1mm of a0] (al)  {\textbf{a:}};
        \node[left = 1mm of b0] (bl1) {\textbf{a[0]:}};
        \node[left = 1mm of b1] (bl2) {\textbf{a[1]:}};
        \node[left = 1mm of p ] (pl)  {\textbf{p:}};
        
        
        \path[draw,-stealth] (b0.south) -- (a0.north);
        \path[draw,-stealth] (b1.south) -- (a3.north);
        \path[draw,-stealth] (p.north) -- (a4.south);
        \end{tikzpicture}
    \end{figure}
\end{frame}

\begin{frame}[fragile]{Расположение в памяти}
    \begin{figure}
        \begin{tikzpicture}[align=center,font=\ttfamily]
        \node[draw,blue,minimum height=5mm, minimum width = 10mm] (a0) {1};
        \node[draw,blue,rectangle,right = 0 of a0,minimum height=5mm, minimum width = 10mm] (a1) {2};
        \node[draw,blue,rectangle,right = 0 of a1,minimum height=5mm, minimum width = 10mm] (a2) {3};
        \node[draw,brown,rectangle,right = 0 of a2,minimum height=5mm, minimum width = 10mm] (a3) {4};
        \node[draw,brown,rectangle,right = 0 of a3,minimum height=5mm, minimum width = 10mm] (a4) {5};
        \node[draw,brown,rectangle,right = 0 of a4,minimum height=5mm, minimum width = 10mm] (a5) {6};
        
        
        \node[draw,blue,rectangle,above = of a0,minimum height=5mm, minimum width = 10mm] (b0) {};  
        \node[draw,brown,rectangle,above = of a3,minimum height=5mm, minimum width = 10mm] (b1) {};
        \node[draw,rectangle,below = of a4,minimum height=5mm, minimum width = 10mm] (p) {};
        
        \node[left = 1mm of a0] (al)  {\textbf{a:}};
        \node[left = 1mm of b0] (bl1) {\textbf{a[0]:}};
        \node[left = 1mm of b1] (bl2) {\textbf{a[1]:}};
        \node[left = 1mm of p ] (pl)  {\textbf{p:}};
        
        
        \path[draw,blue,-stealth] (b0.south) -- (a0.north);
        \path[draw,brown,-stealth] (b1.south) -- (a3.north);
        \path[draw,-stealth] (p.north) -- (a4.south);
        \end{tikzpicture}
    \end{figure}
\end{frame}

\begin{frame}[fragile]{Адресная арифметика}
    \footnotesize
    \begin{columns}
        \begin{column}{0.7\textwidth}
            \begin{cppcode}
                #include <iostream>        
                using namespace std;   
                
                const int DIM1 = 3;
                const int DIM2 = 5;
                
                int array[DIM1][DIM2];
                
                int main() {            
                    for (int i = 0; i < DIM1; i++) {
                        for (int j = 0; j < DIM2; j++) {
                            array[i][j] = (i + 1) * 10 + (j + 1);
                        }
                    }
                    
                    int *ptr = (int *)array;    // так не надо делать ;-)
                    
                    for (int i = 0; i < DIM1; i++) {
                        for (int j = 0; j < DIM2; j++) {
                            cout << *(ptr + i * DIM2 + j) << "\t";
                        }
                        cout << endl;
                    }
                    
                    return 0;
                }
            \end{cppcode}
        \end{column}
        \begin{column}{0.3\textwidth}
            \begin{minted}{bash}
11  12  13  14  15
21  22  23  24  25
31  32  33  34  35
            \end{minted}
        \end{column}
    \end{columns}
    
\end{frame}

\begin{frame}[fragile]{Адресная арифметика}
    \footnotesize
    \begin{columns}
        \begin{column}{0.7\textwidth}
            \begin{cppcode}
                #include <iostream>        
                using namespace std;   
                
                const int DIM1 = 3;
                const int DIM2 = 5;
                
                int array[DIM1][DIM2];
                
                int main() {            
                    for (int i = 0; i < DIM1; i++) {
                        for (int j = 0; j < DIM2; j++) {
                            array[i][j] = (i + 1) * 10 + (j + 1);
                        }
                    }
                    
                    int *ptr = (int *)array;
                    
                    for (int i = 0; i < DIM1 * DIM2; i++) {
                        cout << ptr[i] << "\t";
                    }
                    cout << endl;
                    
                    return 0;
                }
            \end{cppcode}
        \end{column}
        \begin{column}{0.3\textwidth}
            \begin{minted}[breaklines=true]{bash}
11  12  13  14  15  21  22  23  24  25  31  32  33  34  35
            \end{minted}
        \end{column}
    \end{columns}
    
\end{frame}

\begin{frame}[fragile]{Двумерный массив как одномерный}
    \footnotesize
    \begin{columns}
        \begin{column}{0.7\textwidth}
            \begin{cppcode}
                const int DIM1 = 3;
                const int DIM2 = 5;
                
                int ary[DIM1 * DIM2];
                
                int main() {
                    
                    for (int i = 0; i < DIM1; i++) {
                        for (int j = 0; j < DIM2; j++) {
                            *(ary + i * DIM2 + j) = (i + 1) * 10 + (j + 1);
                        }
                    }
                    
                    for (int i = 0; i < DIM1; i++) {
                        for (int j = 0; j < DIM2; j++) {
                            cout << *(ary + i * DIM2 + j) << "\t";
                        }
                        cout << endl;
                    }
                    
                    return 0;
                }
            \end{cppcode}
        \end{column}
        \begin{column}{0.3\textwidth}
            \begin{minted}[breaklines=true]{bash}
11  12  13  14  15  21  22  23  24  25  31  32  33  34  35
            \end{minted}
        \end{column}
    \end{columns}
    
\end{frame}

\section{Динамическое выделение памяти}

\begin{frame}[fragile]{Необходимость динамического выделения памяти}
    Как статическое, так и автоматическое распределение памяти имеют два общих свойства:
    
    \begin{itemize}
        \item Размер переменной/массива должен быть известен во время компиляции.        
        \item Выделение и освобождение памяти происходит автоматически (когда переменная создаётся/уничтожается).
    \end{itemize}
        
    Если нам нужно объявить размер всех переменных во время компиляции, то самое лучшее, что мы можем сделать — это попытаться угадать их максимальный размер, надеясь, что этого будет достаточно:
    \begin{cppcode}
        char name[30]; // будем надеяться, что пользователь введёт имя длиной менее 30 символов!
        Monster monster[30]; // 30 монстров максимум
        Polygon rendering[40000]; // этому 3d rendering-у лучше состоять из менее чем 40 000 полигонов!
    \end{cppcode}
\end{frame}

\begin{frame}[fragile]{Недостатки статического выделения памяти}
    
    \begin{itemize}
        \item теряется память, если массив не используется, или используется не весь;       
        \item статически выделяемая память выделяется из специального хранилища -- \alert{стека}; его размер как правило ограничен 1 МБ памяти, превышение данного значения ведет к завершению программы с ошибкой переполнения стека "stack overflow";
        \item теряется информация если ввод превосходит заранее заданный размер.
    \end{itemize}    
\end{frame}

\begin{frame}[fragile]{Оператор new}
    
    \begin{myDef}
        \alert{Динамическое выделение памяти} — это способ запроса памяти из операционной системы запущенными программами по надобности. Эта память не выделяется из ограниченной памяти стека программы, а из гораздо большего хранилища, управляемого операционной системой — \alert{кучи}. На современных компьютерах размер кучи может составлять гигабайты памяти.
    \end{myDef}
    
    \begin{cppcode}
        int *ptr = new int; // динамически выделяем целочисленную переменную и присваиваем её адрес ptr, чтобы потом иметь доступ к ней
        *ptr = 8; // присваиваем значение 8 только что выделенной памяти        
    \end{cppcode}

    Оператор \mintinline{cpp}{new} возвращает \alert{указатель}, содержащий адрес выделенной памяти.
\end{frame}

\begin{frame}[fragile]{Оператор delete}
    \begin{cppcode}
        // Предположим, что ptr ранее уже был выделен с помощью оператора new
        delete ptr; // возвращаем память, на которую указывал ptr, обратно в операционную систему
        ptr = 0; // делаем ptr нулевым указателем (используйте nullptr вместо 0 в C++11)
    \end{cppcode}
    Хотя может показаться, что мы удаляем переменную, но это не так! Переменная-указатель по-прежнему имеет ту же область видимости, что и раньше, и ей можно присвоить новое значение, как и любой другой переменной.
    
    Обратите внимание, удаление указателя, не указывающего на динамически выделенную память, может привести к проблемам.
\end{frame}

\begin{frame}{Стек и куча}
    \begin{columns}
        \begin{column}{0.5\textwidth}
            \begin{figure}
                \begin{tikzpicture}[align=center,font=\ttfamily]
                \node[draw,minimum height=5mm, minimum width = 10mm] (s0) {};
                \node[draw,rectangle,below = 0 of s0,minimum height=5mm, minimum width = 10mm] (s1) {};
                \node[draw,rectangle,below = 0 of s1,minimum height=5mm, minimum width = 10mm] (s2) {};
                \node[draw,rectangle,below = 0 of s2,minimum height=5mm, minimum width = 10mm] (s3) {};
                \node[draw,rectangle,below = 0 of s3,minimum height=5mm, minimum width = 10mm] (s4) {};
                \node[draw,rectangle,below = 0 of s4,minimum height=5mm, minimum width = 10mm] (s5) {};
                \node[draw,rectangle,below = 0 of s5,minimum height=5mm, minimum width = 10mm] (s6) {};
                \end{tikzpicture}
                \caption{stack}
            \end{figure}
        \end{column}
        \begin{column}{0.5\textwidth}
            \begin{figure}
                \begin{tikzpicture}[align=center,font=\ttfamily]
                \node[draw,minimum height=5mm, minimum width = 10mm] (h0) {};
                \node[draw,rectangle,below right = 0 of h0,minimum height=5mm, minimum width = 10mm] (h1) {};
                \node[draw,rectangle,below = 0 of h0,minimum height=5mm, minimum width = 10mm] (h2) {};
                \node[draw,rectangle,below left = 0 of h0,minimum height=5mm, minimum width = 10mm] (h3) {};
                \node[draw,rectangle,below = 0 of h2,minimum height=5mm, minimum width = 10mm] (h4) {};
                \node[draw,rectangle,below = 0 of h3,minimum height=5mm, minimum width = 10mm] (h5) {};
                \node[draw,rectangle,below = 0 of h1,minimum height=5mm, minimum width = 10mm] (h6) {};
                \end{tikzpicture}
                \caption{heap}
            \end{figure}
        \end{column}
    \end{columns}        
\end{frame}

\begin{frame}[fragile]{Висячие указатели}
    \begin{cppcode}
        #include <iostream>
        
        int main()
        {
            int *ptr = new int; // динамически выделяем целочисленную переменную
            *ptr = 8; // помещаем значение в выделенную ячейку памяти
            
            delete ptr; // возвращаем память обратно в операционную систему. ptr теперь является висячим указателем
            
            std::cout << *ptr; // разыменование висячего указателя приведёт к неожиданным результатам
            delete ptr; // попытка освободить память снова приведёт к неожиданным результатам также
            
            return 0;
        }
    \end{cppcode}
\end{frame}

\begin{frame}[fragile]{Висячие указатели}
    \begin{cppcode}
        #include <iostream>
        
        int main()
        {
            int *ptr = new int; // динамически выделяем целочисленную переменную 
            int *otherPtr = ptr; // otherPtr теперь указывает на ту же самую выделенную память, что и ptr
            
            delete ptr; // возвращаем память обратно в операционную систему. ptr и otherPtr теперь висячие указатели
            ptr = 0; // ptr теперь уже nullptr
            
            // Однако otherPtr по-прежнему является висячим указателем!
            
            return 0;
        }
    \end{cppcode}
\end{frame}


\begin{frame}[fragile]{Нулевые указатели и динамическое выделение памяти}
    Нулевые указатели (указатели со значением 0 или nullptr) особенно полезны в процессе динамического выделения памяти. Их наличие как бы сообщаем нам: «Этому указателю не выделено никакой памяти». А это, в свою очередь, можно использовать для выполнения условного выделения памяти:
    \begin{cppcode}
        // Если ptr-у до сих пор не выделено памяти, то выделяем её
        if (!ptr)
            ptr = new int;
    \end{cppcode}
    Удаление нулевого указателя ни на что не влияет. Таким образом, в следующем нет необходимости:
    \begin{cppcode}
        if (ptr)
            delete ptr;
    \end{cppcode}
    Вместо этого вы можете просто написать:
    \begin{cppcode}
        delete ptr;
    \end{cppcode}
\end{frame}

\begin{frame}[fragile]{Утечка памяти}
    Динамически выделенная память не имеет области видимости, т.е. она остаётся выделенной до тех пор, пока не будет явно освобождена или пока ваша программа не завершит своё выполнение (и операционная система очистит все буфера памяти самостоятельно). Однако указатели, используемые для хранения динамически выделенных адресов памяти, следуют правилам области видимости обычных переменных. Например:
    \begin{cppcode}
        void doSomething(){
            int *ptr = new int;
        }
    \end{cppcode}

   
\end{frame}

\begin{frame}[fragile]{Утечка памяти}
    Хотя утечка памяти может возникнуть и из-за того, что указатель выходит из области видимости, возможны и другие способы, которые могут привести к утечкам памяти. Например, если указателю, хранящему адрес динамически выделенной памяти, присвоить другое значение:
    \begin{cppcode}
        int value = 7;
        int *ptr = new int; // выделяем память
        ptr = &value; // старый адрес утерян - произойдёт утечка памяти
    \end{cppcode}
    
    Корректный способ освобождения памяти:
    \begin{cppcode}
        int value = 7;
        int *ptr = new int; // выделяем память
        delete ptr; // возвращаем память обратно в операционную систему
        ptr = &value; // переприсваиваем указателю адрес value
    \end{cppcode}

    Кроме того, утечка памяти также может произойти и через двойное выделение памяти:
    \begin{cppcode}
        int *ptr = new int;
        ptr = new int; // старый адрес утерян - произойдёт утечка памяти
    \end{cppcode}
    
\end{frame}

\section{Динамические многомерные массивы}

\begin{frame}[fragile]{Создание и уничтожение динамических многомерных массивов}
	\scriptsize 
	\begin{cppcode}
		const int DIM1 = 3;
		const int DIM2 = 5;
		
		int main() {			
			int **array;    // (1)			
			// создание
			array = new int * [DIM1]; // массив указателей (2)
			for (int i = 0; i < DIM1; i++) { // (3)
				array[i] = new int [DIM2]; // инициализация указателей
			}
			
			// работа с массивом
			for (int i = 0; i < DIM1; i++) {
				for (int j = 0; j < DIM2; j++) {
					array[i][j] = (i + 1) * 10 + (j + 1);
				}
			}
			
			for (int i = 0; i < DIM1; i++) {
				for (int j = 0; j < DIM2; j++) {
					cout << array[i][j] << "\t";
				}
				cout << endl;
			}
			
			// уничтожение
			for (int i = 0; i < DIM1; i++) {
				delete [] array[i];
			}
			delete [] array;
			
			return 0;
		}
	\end{cppcode}
\end{frame}

\begin{frame}
	(1) Для доступа к двумерному массиву объявляется переменная \mintinline{cpp}{array} типа указатель на указатель на тип (в данном случае это указатель на указатель на \mintinline{cpp}{int}).
	
	(2) Переменная инициализируется оператором new, который выделяет память для массива указателей на \mintinline{cpp}{int}.
	
	(3) В цикле каждый элемент массива указателей инициализируется оператором new, который выделяет память для массива типа \mintinline{cpp}{int}.
	
	Освобождение памяти происходит строго в обратном порядке: сначала уничтожаются массивы значений типа \mintinline{cpp}{int}, а затем уничтожается массив указателей.
	
	Работа с динамическим многомерным массивом синтаксически полностью совпадает с работой с многомерным C-массивом.
\end{frame}


\begin{frame}[fragile]{Расположение в памяти}
	\begin{figure}
		\begin{tikzpicture}[align=center,font=\ttfamily]
		\node[draw,minimum height=5mm, minimum width = 10mm] (a00) {11};
		\node[draw,rectangle,right = 0 of a00,minimum height=5mm, minimum width = 10mm] (a01) {12};
		\node[draw,rectangle,right = 0 of a01,minimum height=5mm, minimum width = 10mm] (a02) {13};
		\node[draw,rectangle,right = 0 of a02,minimum height=5mm, minimum width = 10mm] (a03) {14};
		\node[draw,rectangle,right = 0 of a03,minimum height=5mm, minimum width = 10mm] (a04) {15};
		
		\node[draw,minimum height=5mm, minimum width = 10mm] (a10) at (0,3) {21};
		\node[draw,rectangle,right = 0 of a10,minimum height=5mm, minimum width = 10mm] (a11) {22};
		\node[draw,rectangle,right = 0 of a11,minimum height=5mm, minimum width = 10mm] (a12) {23};
		\node[draw,rectangle,right = 0 of a12,minimum height=5mm, minimum width = 10mm] (a13) {24};
		\node[draw,rectangle,right = 0 of a13,minimum height=5mm, minimum width = 10mm] (a14) {25};
		
		\node[draw,minimum height=5mm, minimum width = 10mm] (a20) at (4,1) {31};
		\node[draw,rectangle,right = 0 of a20,minimum height=5mm, minimum width = 10mm] (a21) {32};
		\node[draw,rectangle,right = 0 of a21,minimum height=5mm, minimum width = 10mm] (a22) {33};
		\node[draw,rectangle,right = 0 of a22,minimum height=5mm, minimum width = 10mm] (a23) {34};
		\node[draw,rectangle,right = 0 of a23,minimum height=5mm, minimum width = 10mm] (a24) {35};
		
		\node[draw,rectangle,above = of a0,minimum height=5mm, minimum width = 10mm] (b0) {};  
		\node[draw,rectangle,right = 0 of b0,minimum height=5mm, minimum width = 10mm] (b1) {};
		\node[draw,rectangle,right = 0 of b1,minimum height=5mm, minimum width = 10mm] (b2) {};
		
		\node[left = 1mm of a00] (bl0)  {\textbf{a[0]:}};
		\node[left = 1mm of a10] (bl1) {\textbf{a[1]:}};	
		\node[left = 1mm of a20] (bl2) {\textbf{a[2]:}};		
		\node[left = 1mm of b0] (array) {\textbf{array:}};	
		
		\path[draw,-stealth] (b0.south) -- (a00.north);
		\path[draw,-stealth] (b1.north) -- (a10.south);
		\path[draw,-stealth] (b2.east) -- (a20.west);
		
		\end{tikzpicture}
	\end{figure}
\end{frame}

\begin{frame}[fragile]{Двумерный вектор}
	\footnotesize 
	\begin{cppcode}
		const int DIM1 = 3;
		const int DIM2 = 5;
		#include <vector>
		
		using namespace std;
		
		int main() {			
			vector<vector<int>> v;
			
			// заполнение
			for (int i = 0; i < DIM1; i++){
				vector<int> temp;
				v.push_back(temp);
				for (int j = 0; j < DIM2; j++) {
					v.at(i).push_back(
						(i + 1) * 10 + (j + 1));
				}
			}	
					
			for (int i = 0; i < DIM1; i++) {
				for (int j = 0; j < DIM2; j++) {
					cout << v.at(i).at(j) << "\t"; // v[i][j]
				}
				cout << endl;
			}
					
			return 0;
		}
	\end{cppcode}
\end{frame}




% https://ravesli.com/urok-85-dinamicheskoe-vydelenie-pamyati-operatory-new-i-delete/
% https://code-live.ru/post/cpp-array-tutorial-part-2/#_2

%\begin{frame}[t,allowframebreaks]
%    \frametitle{Литературные источники}
%    
%    \nocite{*}
%    \printbibliography
%\end{frame}


\end{document}