% !TeX encoding = UTF-8
% !TeX spellcheck = ru_RU
% !TeX root = lecture2.tex

\documentclass[
    9pt,
    hyperref={pdfencoding=unicode}
    ]{beamer}

\usepackage{polyglossia}   %% загружает пакет многоязыковой вёрстки

\setmainlanguage{russian}  %% устанавливает главный язык документа
\setmainfont{Times New Roman}
\setromanfont{Times New Roman} 
\setmonofont{Fira Code} 
\setsansfont{Arial} 

\PolyglossiaSetup{russian}{indentfirst=true}

\newfontfamily{\cyrillicfont}{Times New Roman}[Ligatures=TeX]
\newfontfamily{\cyrillicfontrm}{Times New Roman}[Ligatures=TeX]
\newfontfamily{\cyrillicfonttt}{Fira Code}[Ligatures=TeX]
\newfontfamily{\cyrillicfontsf}{Arial}[Ligatures=TeX]

\usepackage{wasysym,amssymb}

\usepackage{pgfplots}
\usetikzlibrary{arrows,positioning,shapes}

\usepackage{minted}
\newminted{cpp}{frame=leftline,framerule=1.5pt,framesep=1em,breaklines=true,autogobble=true}

\usetheme{default}
\usecolortheme{seagull}
\useoutertheme[subsection=false]{miniframes}
\setbeamertemplate{mini frames}[tick]

\usepackage{ragged2e}

\usepackage{xcolor}

\makeatletter
\setbeamertemplate{section in head/foot}{%
    \parbox[c][0.33cm][t]{\dimexpr(\textwidth-1.3cm)/\beamer@sectionmax\relax}{%\
        \hyphenpenalty=9999\RaggedRight\fontsize{4}{4}\selectfont\insertsectionhead}}
\makeatother

\author{Воронин Андрей Андреевич}
\title{Лекция 2 – введение в синтаксис языка C++}
\institute{Кафедра прикладной математики и информатики}
\date{\today}
\begin{document}
\titlepage 

\section{Введение в синтаксис}
\begin{frame}{Типы данных}
    \begin{itemize}
        \item<1-> Целочисленные:
        \only<1>{
        \begin{enumerate}
            \item \mintinline{cpp}{char} 
            \item \mintinline{cpp}{short int}
            \item \mintinline{cpp}{int}
            \item \mintinline{cpp}{long int}
            \item \mintinline{cpp}{long long int}
        \end{enumerate}}
        \only<2->{
        \begin{enumerate}
            \item \mintinline{cpp}{int8_t} 
            \item \mintinline{cpp}{int16_t}
            \item \mintinline{cpp}{int32_t}
            \item \mintinline{cpp}{int32_t}
            \item \mintinline{cpp}{int64_t}
        \end{enumerate}}
        Могут быть беззнаковыми (\mintinline{cpp}{unsigned} или \mintinline{cpp}{uint8_t})
        \begin{itemize}
            \item $ -2^{n-1} \ldots (2^{n-1}-1) $ ($n$ -- число бит)
            \item $ 0 \ldots (2^{n}-1) $ (для \mintinline{cpp}{unsigned})
        \end{itemize}
        \item Числа с плавающей точкой:
        \begin{enumerate}
            \item \mintinline{cpp}{float}, 4 байта, 7 значащих цифр.
            \item \mintinline{cpp}{double}, 8 байт, 15 значащих цифр.
        \end{enumerate}
        \item Логический тип данных \mintinline{cpp}{bool}.
        \item Пустой тип \mintinline{cpp}{void}.        
    \end{itemize}
\end{frame}

\begin{frame}{Литералы}
    \begin{itemize}
        \item Целочисленные:
        \begin{enumerate}
            \item \mintinline{cpp}{'a'} -- код буквы \texttt{'a'}, тип \mintinline{cpp}{char}.
            \item \mintinline{cpp}{42} -- все целые по умолчанию типа \mintinline{cpp}{int}.
            \item \mintinline{cpp}{123456789L} -- суффикс \texttt{'L'} соответствует типу \mintinline{cpp}{long}.
            \item \mintinline{cpp}{1703U} -- суффикс \texttt{'U'} соответствует типу \mintinline{cpp}{unsigned int}.
            \item \mintinline{cpp}{1703UL} -- суффикс \texttt{'UL'} соответствует типу \mintinline{cpp}{unsigned long}.
        \end{enumerate}
        \item Числа с плавающей точкой:
        \begin{enumerate}
            \item \mintinline{cpp}{3.14} -- все числа с точкой по умолчанию \mintinline{cpp}{double}.
            \item \mintinline{cpp}{3.1415F} -- суффикс \texttt{'F'} соответствует типу \mintinline{cpp}{float}.
            \item \mintinline{cpp}{3.0E8} -- соответствует  $ 3\cdot10^8 $
        \end{enumerate}
        \item \mintinline{cpp}{true} \mintinline{cpp}{false} -- значения типа \mintinline{cpp}{bool}.
        \item Строки задаются в двойных кавычках: \mintinline{cpp}{"Test string"}.
    \end{itemize}
\end{frame}

\begin{frame}[fragile]{Переменные}
    \begin{itemize}
        \item При определении переменной указывается её тип. При определении можно сразу задать начальное значение (инициализация).
        \begin{cppcode}
            int i = 10;
            short j = 20;
            bool b = false;
            
            unsigned long l = 123321;
            
            double x = 13.5, y = 3.1415;
            float z;
        \end{cppcode}
    \item Нельзя оставлять переменные неинициализированными.
    \item Нельзя создать переменную пустого типа \mintinline{cpp}{void}.
    \end{itemize}
\end{frame}

\begin{frame}[fragile]{Операции}
    \begin{columns}
        \begin{column}{0.5\textwidth}
            \begin{itemize}
                \item Оператор присваивания: \alert{\texttt{=}}.
                \item Арифметические: 
                \begin{enumerate}
                    \item бинарные: \alert{\texttt{+ - * / \%}},
                    \item унарные: \alert{\texttt{++ --}}.
                \end{enumerate}
                \item Логические:
                \begin{enumerate}
                    \item бинарные: \alert{\texttt{\&\& ||}}
                    \item унарные: \alert{\texttt{!}}.
                \end{enumerate}
                \item Сравнения: \alert{\texttt{== != > < >= <=}}.
                \item Приведения типов: \alert{\texttt{(type)}}.
                \item Сокращенные версии бинарных операторов: \alert{\texttt{+= -= *= /= \%=}}.
            \end{itemize}
        \end{column}
        \begin{column}{0.5\textwidth}  %%<--- here
            \begin{cppcode}
                int i = 10;
                i = (20 * 3) % 7;
                
                int k = i++;
                int l = --i;
                
                bool b = !(k == 1);
                
                b = (a == 0) || (1 / a < 1);
                
                double d = 3.1415;
                float f = (int) d;
                
                // d = d * (i + k);
                d *= i + k;
            \end{cppcode}
        \end{column}
    \end{columns}
\end{frame}

\begin{frame}{Целочисленные типы в C++}
    Все целочисленные типы (кроме \texttt{char}) являются знаковыми. Беззнаковые версии типов определяется с ключевым словом \texttt{unsigned}, например: \texttt{unsigned short int}, \texttt{unsigned int} или \texttt{unsigned long int}. Для симметрии в языке предусмотрено явное указание того, что тип является знаковым — ключевое слово \texttt{signed} (используется редко).
    
    Более того, C++ допускает использование следующих сокращений:
    \begin{itemize}
        \item \texttt{unsigned} вместо \texttt{unsigned int},
        \item \texttt{short} вместо \texttt{short int},
        \item \texttt{long} вместо \texttt{long int}.
    \end{itemize}

    \textbf{Операции инкремента и декремента:}
\end{frame}

\begin{frame}[fragile]{Операции инкремента и декремента}
    \begin{cppcode}
        int a = 10; // a = 10
        int b = ++a; // префиксный инкремент возвращает новое значение => b = 11 и a = 11
        int c = a++; // постфиксный инкремент возвращает старое значение => с = 11 и a = 12
    \end{cppcode}
\end{frame}

\begin{frame}[fragile]{Преобразования типов в операторах}
    Выражениям так же как и значениям в C++ приписывается некоторые типы. Например, если \texttt{a} и \texttt{b} — это переменные типа \texttt{int}, то выражения \texttt{(a + b)}, \texttt{(a - b)}, \texttt{(a * b)} и \texttt{(a / b)} тоже будут иметь тип \texttt{int}.
    
    Важно всегда понимать, какой тип у выражения, которое вы написали в программе. Давайте проиллюстрируем это на следующем примере:
    
    \begin{cppcode}
        int a = 20;
        int b = 50;
        double d = a / b;  // d = 0, оба аргумента целочисленные, а значит деление целочисленное 
    \end{cppcode}

    Для того чтобы исправить эту ситуацию хотя бы один из аргументов оператора деления должен иметь типа double. Этого можно добиться при помощи уже известного нам оператора приведения типов:
    \begin{cppcode}
        double d = (double)a / b;  // d = 0.4
    \end{cppcode}

    Почему это сработало? Дело в том, что операторы для встроенных типов C++ \emph{всегда} работают с \emph{одинаковыми типами} аргументов. Если аргументы имеют разные типы, то происходит преобразование типов (promotion).
\end{frame}

\begin{frame}{Правило преобразования встроенных типов в операторах}
    Рассмотрим выражение \mintinline{cpp}{(a + b)}, где вместо \texttt{'+'} может стоять любой другой подходящий оператор.
    \begin{enumerate}
        \item Если один из аргументов имеет числовой тип с плавающей точкой, то второй аргумент приводится к этому типу (например, при сложении \texttt{double} и \texttt{int} значение типа \texttt{int} приводится к \texttt{double}).
        \item Если оба аргумента имеют числовой тип с плавающей точкой, то выбирается наибольший из этих типов (например, при сложении \texttt{double} и \texttt{float} значение типа \texttt{float} приводится к \texttt{double}).
        \item Если оба аргумента целочисленные, но их типы меньше \texttt{int}, то оба аргумента приводятся к типу \texttt{int} (например, при сложении двух значений типа \texttt{char} они оба сначала приводятся к \texttt{int}).
        \item Если оба аргумента целочисленные, то аргумент с меньшим типом приводится к типу второго аргумента (например, при сложении \texttt{long} и \texttt{int} значение типа \texttt{int} приводится к \texttt{long}).
        \item Если оба аргумента целочисленные и имеют тип одного размера, то предпочтение отдаётся беззнаковому типу (например, при сложении \texttt{int} и \texttt{unsigned int} значение типа \texttt{int} приводится к \texttt{unsigned int}).
    \end{enumerate}
\end{frame}

\begin{frame}[fragile]{Следствия неявных преобразований типов}
    \begin{enumerate}
        \item Следите за тем, какие типы участвуют в выражении, от этого может зависеть его значение.
        \item Не стоит использовать целочисленные типы меньше \texttt{int} в арифметических выражениях, они всё равно будут приведены к \texttt{int}.
        \item \alert{Не стоит смешивать \texttt{unsigned} и \texttt{signed}} типы в одном выражении, это может привести к неприятным последствиям.
    \end{enumerate}

    Для иллюстрации последнего следствия давайте рассмотрим следующий пример:
    \begin{cppcode}
        unsigned from = 100;
        unsigned to = 0;
        for (int i = from; i >= to; --i) {  ....  }
    \end{cppcode}    
\end{frame}

\begin{frame}[fragile]{Инструкции}
    \begin{itemize}
        \item Выполнение состоит из последовательности \emph{инструкций}.
        \item Инструкции выполняются одна за другой.
        \item Порядок вычислений внутри инструкций неопределен.
        \begin{cppcode}
            /* unspecified behavior */
            int i = 10;
            i = (i+=6) + (i * 4);
        \end{cppcode}
        \item Блоки имеют вложенную область видимости:
        \begin{cppcode}
            int k = 10;
            {
                int k = 5 * i; // не видна за пределами блока
                i = (k += 5) + 5;
            }
            k = k + 1;
        \end{cppcode}
    \end{itemize}    
\end{frame}

\begin{frame}[fragile]{Условные операторы}
    \begin{itemize}
        \item Оператор \mintinline{cpp}{if}:
        \begin{cppcode}
            int d = b * b - 4 * a * c;
            if (d > 0)
            {
                roots = 2;
            }
            else if (d == 0)
            {
                roots = 1;
            }
            else
            {
                roots = 0;
            }
        \end{cppcode}
        \item Тернарный условный оператор:
        \begin{cppcode}
            int roots = 0;
            if (d >= 0)
                roots = (d > 0) ? 2 : 1;
        \end{cppcode}
    \end{itemize}
\end{frame}

\begin{frame}[fragile]{Циклы}
    \begin{itemize}
        \item Цикл \mintinline{cpp}{while}:
        \begin{cppcode}
            int squares = 0;
            int k = 0;
            while (k < 0)
            {
                squares += k * k;
                k = k + 1;
            }
        \end{cppcode}
        \item Цикл \mintinline{cpp}{for}:
        \begin{cppcode}
            for (int k = 0; k < 10; k = k + 1)
            {
                squares += k * k;
            }
        \end{cppcode}
        \item Для выхода из цикла используется оператор \mintinline{cpp}{break}.
    \end{itemize}
\end{frame}

\begin{frame}[fragile]{Цикл do-while}
    В С++ существует вариация цикла \texttt{while}, которая называется \texttt{do-while}. В отличие от обычного \texttt{while} в \texttt{do-while} условие проверяется не до, а \emph{после} итерации. Т.е. такой цикл всегда имеет хотя бы одну итерацию.
    
    \begin{cppcode}
        int i = 10;
        int sum = 0;
        while (i < 10) {
            sum += i;
        }
        // sum = 0
    \end{cppcode}
    
    \vspace{1em}
    \begin{cppcode}
        int i = 10;
        int sum = 0;
        do {
            sum += i;
        } while(i < 10);
        // sum = 10
    \end{cppcode}
\end{frame}

\begin{frame}[fragile]{Управление циклами}
    Ранее был упомянут оператор \mintinline{cpp}{break}, который используется для выхода из цикла. Рассмотрим его действие на следующем примере.
    \begin{cppcode}
        int a = 323;
        int b = 2;
        while ( b <= a ) { 
            if ( a % b == 0 )
            break; // выйти из цикла
            b = b + 1;
        }
    \end{cppcode}
    
    В данном случае \texttt{b} будет равен $ 17 $, т.к. $ 323 = 17 \times 19 $.
    
    \vspace{1em}
    Ещё один оператор, который можно использовать с циклами — это оператор \mintinline{cpp}{continue}. Оператор \mintinline{cpp}{continue} прерывает текущую итерацию цикла и переходит к следующей.
    \begin{cppcode}
        int sum = 0;
        for ( int i = 1; i <= 100; ++i ) {
            if ( (i % 17 == 0) || (i % 19 == 0) )
            continue; // перейти к следующей итерации
            sum += i;
        }
    \end{cppcode}
\end{frame}

\begin{frame}[fragile]{Потоки ввода и вывода}
        Вывод различных типов в С++
        \begin{cppcode}
            #include <iostream>
            
            int main()
            {
                int i = 42;
                double d = 3.14;
                const char *s = "C-style string";
                
                std::cout << "This is an integer " << i << "\n";
                std::cout << "This is a double " << d << "\n";
                std::cout << "This is a \"" << s << "\"\n";
                
                return 0;
            }
        \end{cppcode}
    
\end{frame}

\begin{frame}[fragile]{Потоки ввода и вывода}
      
    Вывод и ввод в С++ 
    \begin{cppcode}
        #include <iostream>
        
        int main()
        {
            int i = 42;
            double d = 3.14;
            
            std::cout << "Enter an integer and a double:\n";
            std::cin >> i >> d;
            std::cout << "Your input is " << i << ", " << d << "\n";
            
            return 0;
        }
    \end{cppcode}   
\end{frame}

\begin{frame}[fragile]{Посимвольный ввод}
    \begin{cppcode}
        char c = '\0';
        while (cin.get(c)) { // на каждой итерации считываем один символ в переменную c
            /* здесь можно пользоваться значением прочитанным в переменную c */
            if (c != 'a')
            cout << c; // выводим символ, если он не равен 'a'
        }
    \end{cppcode}   
\end{frame}

\begin{frame}[fragile]{Функции}
    \begin{itemize}
        \item В сигнатуре функции указывается тип возвращаемого
        значений и типы параметров.
        \item Ключевое слово return возвращает значение.

        \begin{cppcode}
            double square ( double x ) {
                return x * x ;
            }

        \end{cppcode}
        \item Переменные, определённые внутри функций, — \emph{локальные}.
        \item Функция может возвращать \mintinline{cpp}{void}
        \item Параметры передаются по значению (копируются).
        \begin{cppcode}
            void strange ( double x , double y ) {
                x = y ;
            }

        \end{cppcode}
    \end{itemize}
\end{frame}

\begin{frame}[fragile]{Макросы}
    \begin{itemize}
        \item Макросами в C++ называют инструкции препроцессора.
        \item Препроцессор C++ является самостоятельным языком,
        работающим с произвольными строками.

        \item Макросы можно использовать для определения функций:
        \begin{cppcode}
            int max1 ( int x , int y ) {
                return x > y ? x : y ;
            }
            #define max2 (x , y ) x > y ? x : y
            a = b + max2 (c , d ); // b + c > d ? c : d;
        \end{cppcode}
        \item Препроцессор “не знает” про синтаксис C++
    \end{itemize}
\end{frame}

\begin{frame}[fragile]{Макросы}
    \begin{itemize}
        \item Параметры макросов нужно оборачивать в скобки:
        \begin{cppcode}
            #define max3 (x , y ) (( x ) > ( y ) ? (x ) : ( y ))
        \end{cppcode}
        \item Это не избавляет от всех проблем:
        \begin{cppcode}
            int a = 1;
            int b = 1;
            int c = max3 (++ a , b );
            // c = ((++a) > (b) ? (++a) : (b))
        \end{cppcode}
        \item Определять функции через макросы — плохая идея.
        \item Макросы можно использовать для условной компиляции:
        \begin{cppcode}
            #ifdef DEBUG
            // дополнительные проверки
            #endif

        \end{cppcode}
    \end{itemize}
\end{frame}

\begin{frame}[fragile]{Макросы}
    Предположим, что в программе определён макрос sqr.
    \begin{cppcode}
        #define sqr(x) x * x
    \end{cppcode}
    
    Какое значение будет иметь следующее выражение?
    \begin{cppcode}
        sqr(3 + 0)
    \end{cppcode}
\end{frame}

\begin{frame}[fragile]{Ввод-вывод}
    \begin{itemize}
        \item Будем использовать библиотеку \mintinline{cpp}{<iostream>}
        \begin{cppcode}
            #include <iostream>
            using namespace std ;

        \end{cppcode}
        \item Ввод:
        \begin{cppcode}
            int a = 0;
            int b = 0;
            cin >> a >> b ;
            
        \end{cppcode}
        \item Вывод:
        \begin{cppcode}
            cout << " a + b = " << ( a + b ) << endl;
        \end{cppcode}
    \end{itemize}
\end{frame}

\begin{frame}[fragile]{Простая программа}
    \begin{cppcode}
        #include <iostream>
        using namespace std ;
        int main ()
        {
            int a = 0;
            int b = 0;
            cout << " Enter a and b : " ;
            cin >> a >> b ;
            cout << " a + b = " << ( a + b ) << endl ;
            return 0;
        }
    \end{cppcode}
\end{frame}


\end{document}